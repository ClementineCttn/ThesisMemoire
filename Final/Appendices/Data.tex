%\chapter{Datasets}{Données} % Chapter title
\chapter{Données}

\label{app:data} % For referencing the chapter elsewhere, use \autoref{ch:name} 

%----------------------------------------------------------------------------------------


\headercit{}{}{}



This appendix lists and describes the different open datasets created and used in the thesis.

when possible, specify data citation (ex. traffic data : TransportationEquilibrium paper) ; try to put all on dataverse ; laius sur dataverse, partage des données etc.

\textit{Les données comme domaine de connaissance propre : décrire opération de collecte des données et de construction des jeux.}

\section{Grand Paris Traffic Data}{Données de Traffic du Grand Paris}

% données syntadin : checker la licence




\section{US Gaz Prices}{Prix de l'Essence aux Etats-Unis}




\section{European Topological Road Network}{Réseau Routier Européen}




\section{French Freeway Dynamical Network}{Réseau Dynamique des Autoroutes Françaises}

\comment{Merger avec la base bassin parisien de Florent, faire un data paper.}


\section{Interviews}{Interviews}

\label{app:sec:interviews}

% laius sur pourquoi données "quali" devraient pas être plus dispo (quand accord intervié) ; outils idem ex. git, dissocié quanti : cf exemple galère excel notes ridicule, refus systématique et catégorique d'une alternative stable et fiable...

\comment{
Possible interview in Guandong : Zhuhai Planning Bureau (people at the workshop) ; Hong Kong Transportation authority (see demand in name of Medium : how to proceed ?) : easy for english ?
}





