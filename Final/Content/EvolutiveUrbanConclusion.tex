



%----------------------------------------------------------------------------------------

\newpage


\section*{Chapter Conclusion}{Conclusion du Chapitre}


La notion de co-évolution, qui était jusqu'ici relativement conceptuelle, apparaît sous de multiples angles nouveaux complémentaires. On comprend mieux son rôle prépondérant au coeur de la Théorie Evolutive : celle-ci sera également centrale pour la construction théorique que nous élaborerons en~\ref{sec:theory}. En effet, des interdépendances fortes peuvent se traduire par des corrélations locales variables, c'est à dire une non-stationnarité spatiale, induite d'une part par les motifs locaux correspondant à une régime d'interaction donné, dont nous avons pu capturer les manifestations statiques en section~\ref{sec:staticcorrelations}, d'autre part par le caractère multi-scalaire des processus impliqués que nous avons également montré, et donc par les interactions à grande échelle et portée entre les différentes entités territoriales, que nous avons illustré sur un cas simple par le modèle d'interaction étudié en~\ref{sec:interactiongibrat}, qui a déjà pu permettre de révéler indirectement des effets de réseaux dans les systèmes de villes. On a également éclairé une approche dynamique de la co-évolution, en montrant la complexité potentielle de la structure des relations causales dans le cas d'un modèle de morphogenèse urbaine simple. La méthodologie développée s'est montrée également efficace sur les données réelles de l'Afrique du Sud sur le temps long, permettant de découvrir un effet des politiques de ségrégation au second ordre sur la co-évolution elle-même. La question de la non-stationnarité et de la non-ergodicité dans les systèmes urbains est cruciale mais très peu comprise, et nous l'avons à peine effleurée. Dans notre cas, l'aspect le plus important de celle-ci pour la construction des modèles est son implication pour les échelles considérées, et les hypothèses d'équilibre ou de stochasticité correspondantes. On y reviendra par un point de vue différent en Chapitre~\ref{ch:morphogenesis}. Nous proposons pour l'instant de renforcer l'épaisseur thématique des relations considérées : on a en effet pour l'instant seulement étudié des variables très simples (distribution de la population et propriétés du réseau) à certaines échelles seulement. On étudiera ainsi dans le prochain Chapitre~\ref{ch:micro} des ontologies et échelles sur des cas d'étude plus exotiques.



\stars

