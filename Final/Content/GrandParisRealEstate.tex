



%----------------------------------------------------------------------------------------

\newpage



\section{Real Estate Transactions}{Transactions immobilières et Grand Paris}

\label{sec:grandparisrealestate}


%----------------------------------------------------------------------------------------



%%%%%%%%%%%%%
\subsection{Context}{Contexte}






%%%%%%%%%%%%%%%
\subsection{Discussion}{Discussion}
%%%%%%%%%%%%%%%


%%%%%%%%%%%%%%%
\subsubsection{Spatio-temporal diffusion}{Diffusion spatio-temporelle}

% STARMA, ondes, ergodicité etc.

\bpar{
The application of our approach must be lead carefully regarding the choice of scales, processes and objects of study. Typically, it will be not adapted to the quantification of spatio-temporal processes for which the temporal scale of diffusion if of the same order than the estimation window, as our stationarity assumption here stays basic. We could propose to proceed to estimations on moving windows but it would then require the elaboration of a spatial correspondence technique to follow the propagation of phenomena. An example of concrete application that would have a strong thematic impact would be a characterization of a fundamental component of the Evolutive Urban Theory that is the hierarchical diffusion of innovation between cities~\cite{pumain2010theorie}. This would be done by analyzing potential spatio-temporal dynamics of patents classifications such as the one introduced by~\cite{10.1371/journal.pone.0176310}. We also underline that these are rather open methodological questions, for which a concretisation is the potential link between the non-ergodic properties of urban systems~\cite{pumain2012urban} and a wave-based characterization of these processes.
}{
L'application de notre approche doit être menée précautionneusement concernant le choix des  échelles, processus et objets d'étude. Typiquement, elle ne sera pas du tout adaptée à la quantification de processus spatio-temporels dont l'échelle temporelle de diffusion est de l'ordre de celle de la fenêtre d'estimation : l'hypothèse de stationnarité est basique. On peut proposer de procéder à des estimations par fenêtres glissantes, mais il faudrait ensuite élaborer une technique de correspondance spatiale pour traquer la propagation des phénomènes. Un exemple d'application concrète à l'impact thématique fort serait une caractérisation d'une composante fondamentale de la Théorie Evolutive des Villes, la diffusion hiérarchique de l'innovation entre les villes~\cite{pumain2010theorie}, en analysant les potentielles dynamiques spatio-temporelles des classifications de brevets comme celle introduite par~\cite{10.1371/journal.pone.0176310}. Il faut noter toutefois qu'il s'agit de questions méthodologiques relativement ouvertes, dont une des manifestations est le lien potentiel entre le caractère non-ergodique des systèmes urbains~\cite{pumain2012urban} et une caractérisation ondulatoire de ces processus.
}



%%%%%%%%%%%%%%%
\subsubsection{Geographically Weighted Regression}{Regression Géographique Pondérée}

% lien avec GWR ?

\bpar{
An other direction for developments and potential applications can be found when going to a more local scale, by exploring an hybridation with Geographically Weighted Regression techniques~\cite{brunsdon1998geographically}. The determination by cross-validation of Akaike criterion of an optimal spatial scale for the performance of these models, as done by~\cite{2017arXiv170607467R} in a multi-modeling fashion, could be adapted in our case to determine a local optimal scale on which lagged correlations would be the most significant, what would allow to tackle the question of non-stationarity by a mostly spatial approach.
}{
Une autre direction de développement et d'applications potentiels se révèle en se tournant vers l'échelle plus locale, et d'explorer une hybridation avec les techniques de Regression Géographique Pondérée~\cite{brunsdon1998geographically}. La détermination par validation croisée ou Critère d'Akaike d'une portée spatiale optimale pour la performance de ce type de modèles pourrait être adaptée dans notre cas pour déterminer une échelle locale optimale sur laquelle les correlations retardées sont les plus significatives, ce qui permettrait de s'extraire du problème de la non-stationnarité prioritairement par l'aspect spatial.
}

\stars




%%%%%%%%%%%%%%%
%\subsection{Conclusion}{Conclusion}
%%%%%%%%%%%%%%%

%We have introduced a generic method of Granger causality on territorial spatio-temporal data, and shown its potentialities and operational nature with synthetic data and on a real case. We postulate that the simple methodological apparel is am asset for a certain level of generality, but that the application to complex case studies exhibiting circular causalities demonstrate the high potential to contribute to the understanding of dynamics for this type of co-evolutive systems.

%Nous avons proposé une méthode générique de causalité de Granger sur des données territoriales, puis démontré sa potentialité et son caractère opérationnel sur données synthétiques et un cas réel. Nous postulons que l'appareillage méthodologique simple est un atout pour une certaine généralité, mais que l'application à ces cas complexes présentant des causalités circulaires a un fort potentiel de contribution à la compréhension des dynamiques de ce type de systèmes co-évolutifs.









