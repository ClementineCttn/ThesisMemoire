




\section[Transportation Governance Modeling][Gouvernance du Système de Transport]{Transportation Governance Modeling}{Modélisation de la Gouvernance du Système de Transport} 

\label{sec:lutetia}


%----------------------------------------------------------------------------------------


\bpar{
This part makes a step further towards more complex models. A toy-model introducing governance processes is described. Such exploration logically enters our theoretical framework to try to validate the network necessity assumption : if non-linear necessary processes are highlighted and validated against stylized facts, it argues towards the validation of this assumption. 
}{
Cette section fait un pas supplémentaire vers des modèles plus complexes. Un modèle jouet incluant des processus de gouvernance est décrit. Cette exploration répond de manière logique à notre cadre théorique et aux études précédentes, en particulier pour essayer de valider l'hypothèse de nécessité des réseaux : si des processus non-linéaires sont montrés nécessaires pour la validation sur des faits stylisés, cela pousse à argumenter pour sa validité.
}


%----------------------------------------------------------------------------------------


%%%%%%%%%%%%%%%%%%%%
\subsection{Context}{Contexte}





%%%%%%%%%%%%%%%%%%%%
\subsection{Taking Governance into account in Network Production Processes : The Lutecia Model}{Le Modèle Lutecia}




\comment{(Florent) on DC module : c'est à dire ? comparer quoi avec qui ?}

\comment{(Florent) Implémentation : développer les aspects méthodo ``techniques'' ce n'est pas sale, au contraire}






%%%%%%%%%%%%%%%%%%%%
\subsection[Application][Application]{Application to Pearl River Delta}{Application au Delta de la Rivière des Perles}












