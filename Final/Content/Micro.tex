



% Chapter 

%\chapter{Interactions at the Micro Level}{Interactions au niveau microscopique} % Chapter title

\chapter{Echelles et Ontologies}

\label{ch:micro} % For referencing the chapter elsewhere, use \autoref{ch:name} 

%----------------------------------------------------------------------------------------


%\headercit{}{}{}


\bigskip



La richesse des interactions entre réseaux et territoires, développée dans le Chapitre~\ref{ch:thematic}, est que celle-ci se produisent à différentes échelles, entre ces échelles, et par des intermédiaires très variés, au sens des agents ou structures impliquées mais aussi de leur caractéristiques, ceux-ci allant de la congestion des réseaux aux dynamiques sur le temps long en passant par les re-localisations des activités par exemple. Le cas de Zhuhai développé en~\ref{sec:casestudies} illustre la complexité d'une trajectoire locale et régionale, d'une bifurcation politique induisant l'instauration de la Zone Economique Spéciale par \noun{Xi Jinping} conditionnée à une bifurcation historique bien plus ancienne liée à la colonisation européenne qui a conduit à l'existence de Macao\comment[FL]{mal dit}, à une bifurcation géographique\comment[FL]{trop simple, toutes les bifurcations sont a la fois geographique, historique, etc.} en terme d'accessibilité régionale et une nouvelle position centrale de la ville dans la Mega-city Region du Delta de la Rivière des Perles. Nous avons dans le chapitre précédent étudié empiriquement les manifestations morphologiques des interactions à l'échelle mesoscopique\comment[FL]{c'est dans ce chapitre qu'il faut discuter de ces echelles, pas avant}, mais également mis en évidence des effets de structure à cette même échelle sur un temps long dans le cas de l'Afrique du Sud. Quelle échelle minimale est-il pertinent de considérer, autrement dit l'étude de l'échelle microscopique peut-elle nous apporter de l'information ? Et peut-on clarifier certaines ontologies, ou au moins un certain degré de précision ou de complexité requis dans celles-ci ? Ce chapitre cherche à répondre à ces interrogations par le biais d'études empiriques. Ainsi, nous tentons de préciser itérativement la structure des modèles futurs, mais aussi leur non-structure.


Dans une première section~\ref{sec:transportationequilibrium}, nous explorons empiriquement un jeu de données à l'échelle microscopique sur le trafic routier en Ile-de-France, en ayant notamment à l'esprit la notion d'équilibre des flux de trafic qui est une hypothèse particulièrement répandue dans la modélisation du trafic. Nous démontrons que cet équilibre n'a aucun fondement empirique\comment[FL]{et alors ? en quoi est-ce problematique ?}, et que les trajectoires microscopiques du système sont chaotiques. Cela nous permettra d'une part de conforter nos choix épistémologiques de modèle loin de l'équilibre typique d'une appréhension de la complexité, d'autre part de confirmer que cette échelle n'est pas pertinente. Nous continuons sur le traffic routier dans une deuxième section~\ref{sec:energyprice}, en nous concentrons sur la composante du prix de transport via le proxy du prix de vente du carburant, et ces liens potentiels avec les caractéristiques socio-économiques des territoires, dans le cas des Etats-Unis avec une granularité spatiale au comté et temporelle à la journée. Nous obtenons le résultat assez inattendu des deux échelles endogènes proprement définies, correspondant aux échelles mesoscopique et macroscopique, mais aussi la mise en évidence de la superposition de processus de gouvernance à des processus locaux. Enfin, la dernière section~\ref{sec:grandparisrealestate} applique la méthode d'identification de causalités développée en~\ref{sec:causalityregimes} au différents projets de transport du Grand Paris et démontre des potentiels effets d'annonce des projets de transport sur la croissance de la population, confirmant la pertinence d'une échelle d'agrégation au moins mesoscopique et de se concentrer sur des variables territoriales relativement basiques.



\stars


\bpar{
\textit{This chapter is entirely adapted from diverse papers: section \ref{} was published}
}{
\textit{Ce chapitre est entièrement adapté de divers articles: la section~\ref{sec:transportationequilibrium} a été publiée en anglais comme \cite{raimbault2017investigating}; la section~\ref{sec:energyprice} également en anglais en collaboration avec \noun{A. Bergeaud} comme \cite{raimbault2017cost}; la section \ref{sec:grandparisrealestate} correspond à la partie d'application de \cite{raimbault2017identification}.}
}




%----------------------------------------------------------------------------------------




%%%%%%%%%%%%%%%%%%%
\section{Territorial representations}{Représentation territoriales}


% very short section on what to put in a territorial representation of an urban system ; intrinsic dimension ; link with portugali semantic information ? ; why the two next studies are interesting and why they open the subject.

% NOTE : why this need of balance in plan ? makes no sense, specifically regarding order/disorder.


\subsection{Representation of territorial systems}{Representations de systèmes territoriaux}



\subsection{Intrinsic dimension}{Dimension intrinsèque d'un système territorial}



\subsection{Extend ontologies}{Etendre les ontologies}








