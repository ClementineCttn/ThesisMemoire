



%----------------------------------------------------------------------------------------

\newpage

\section{Modeling Interactions}{Modéliser les Interactions}
\label{sec:modelingsa}


%----------------------------------------------------------------------------------------





% -- from algoSR --

%\comment{(Florent) c'est différent de la littérature empirique sur ces interactions}

%\paragraph{Land-Use Transportation Interaction Models}{Modèles LUTI}


%\bpar{
%A wide class of models that have been developed essentially for planning purposes, which are the so-called Land-use Transportation Interaction Models, is a first type answering our research question. See diverse reviews \cite{chang2006models}, \cite{iacono2008models} and \cite{wegener2004land} to get an idea of the heterogeneity of included approaches, that exist for more than 30 years. Recent models with diverse refinements are still developed today, such as \cite{delons:hal-00319087} which includes housing market for Paris area. Diverse aspects of the same system can be translated into many models (as \eg \cite{wegener1991one}), and traffic, residential and employment dynamics, resulting land-use evolution, influenced also by a static transportation network, are generally taken into account.
%}{
%Une large classe de modèle développés essentiellement dans des objectifs de planification, \comment{(Florent) c'est un aspect à part entière mais à mon avis pas dès le début} les modèles d'interaction entre transport de usage du sol, sont un premier type pouvant rentrer dans notre problématique. Voir les diverses revues~\cite{chang2006models},~\cite{iacono2008models} et~\cite{wegener2004land} pour avoir un aperçu de l'hétérogénéité des approches incluses, qui existent depuis plus de 30 ans.\comment{(Florent) seventies}  Des versions récentes avec divers raffinements sont toujours développés aujourd'hui, comme~\cite{delons:hal-00319087} qui inclut le marché immobilier pour la région parisienne. Différents aspects du même système peuvent être traduits par divers modèles (comme e.g. \cite{wegener1991one}), et le traffic, les dynamiques résidentielles et d'emploi, l'évolution de l'usage du sol en découlant, influencée aussi par un réseau de transport statique, sont généralement pris en compte.  \comment{(Florent) description}
%}


%\paragraph{Network Growth Approaches}{Approches de Croissance de Réseau}


%\bpar{
%On the contrary, many economic literature has done the opposite of previous models, i.e. trying to reproduce network growth given assumptions on the urban landscape, as reviewed in \cite{zhang2007economics}. In~\cite{xie2009modeling}, economic empirical studies are positioned within other network growth approaches, such as work by physicists proposing model of geometrical network growth \cite{barthelemy2008modeling}. Analogy with biological networks was also done, reproducing typical robustness properties of transportation networks \cite{tero2010rules}.
%}{
%A l'opposé de nombreux travaux ont pris la logique inverse, i.e. essayent de reproduire la croissance du réseau étant donné des hypothèses sur l'environnement urbain, \comment{(Florent)  sur leurs dynamique, ou bien leur structure, figée ?} comme résumé dans~\cite{zhang2007economics}. Dans~\cite{xie2009modeling}, les travaux économiques empiriques sont positionnés parmi les autres approches de la croissance des réseaux, comme des travaux de physiciens proposant des modèles de croissance géométrique locale~\cite{barthelemy2008modeling}. L'analogie avec la biologie a également déjà été faite, permettant de reproduire les propriétés typiques de robustesse des réseaux de transport~\cite{tero2010rules}. \comment{(Florent) est ce que cela permet de ``valider'' l'analogie ?}
%}


%\paragraph{Hybrid Approaches}{Approches hybrides}


%\bpar{
%Fewer approaches coupling urban growth and network growth can be found in the literature. \cite{barthelemy2009co} couples density evolution with network growth in a toy model. In~\cite{raimbault2014hybrid}, a simple Cellular Automaton coupled with an evolutive network reproduces stylized facts of human settlements described by Le Corbusier. At a smaller scale, \cite{achibet2014model} proposes a model of co-evolution between roads and buildings, following geometrical rules. These approaches stay however limited and rare.
%}{
%Peu de travaux couplant croissance urbaine et croissance du réseau sont disponibles dans la littérature. \cite{barthelemy2009co} couple l'évolution de la densité avec la croissance du réseau dans un modèle jouet. Dans~\cite{raimbault2014hybrid}, un automate cellulaire simple couplé à un réseau évolutif reproduit les faits stylisés des Etablissements Humains décrits par Le Corbusier. \comment{(Florent) à développer là c'est un peu sec ; je n'ai aucune idée de ce qui a pu concrètement être reproduit ou pas.} A une plus petite échelle,~\cite{achibet2014model} propose un modèle de co-évolution entre routes et bâtiments, en suivant des règles géométriques. Ces approches restent cependant limitées et rares.
%}







\subsection{Modeling in Quantitative Geography}{Modélisation en Géographie Quantitative}

% brief reference to the history of TQG ; history of modeling.
%  note : history of future of TQG, London september 2016

% \cite{pumain2002role} : emergence of TQG

\todo{épsitémologie des modèles équilibre/hors équilibre (pas faut dans positionnement épistémo, le faire ici}


\bpar{
Modeling in Theoretical and Quantitative Geography (TQG), and more generally in Social Science, has a long history on which we can not go further than a general context. \noun{Cuyala} does in~\cite{cuyala2014analyse} an analysis of the spatio-temporal development of French speaking TQG movement and underlines the emergence of the discipline as the combination between quantitative analysis (e.g. spatial analysis or modeling and simulation practices) and theoretical constructions, an integration of both allowing the construction of theories from empirical stylized facts that yield theoretical hypothesis to be tested on empirical data. These approach were born under the influence of the \emph{new geography} in Anglo-saxon countries and Sweden. A broad history of the genesis of models of simulation in geography is done by \noun{Rey} in~\cite{rey2015plateforme} with a particular emphasis on the notion of validation of models. The use of computation for simulation of models is anterior to the introduction of paradigms of complexity, coming back to \noun{H{\"a}gerstrand} and \noun{Forrester}, pioneers of spatial economic models inspired by Cybernetics. With the increase of computational possibilities epistemological transformations have also occurred, with the apparition of explicative models as experimental tools. \noun{Rey} compares the dynamism of seventies when computation centers were opened to geographers to the democratization of High Performance Computing (transparent grid computing, see~\cite{schmitt2014half} for an exemple of the possibilities offered in terms of model validation and calibration, decreasing the computational time from 30 years to one week), that is also accompanied by an evolution of modeling practices~\cite{banos2013pour} and techniques~\cite{10.1371/journal.pone.0138212}. Modeling (in particular computational models of simulation) is seen by many as a fundamental building brick of knowledge : \cite{livet2010} recalls the combination of empirical, conceptual (theoretical) and modeling domains with constructive feedbacks between each. A model can be an exploration tool to test assumptions, an empirical tool to validate a theory against datasets, an explicative tool to reveal causalities (and thus internal processes of a system), a constructive tool to iteratively build a theory with an iterative construction of an associated model. These are example among others : \noun{Varenne} proposes in~\cite{varenne2010simulations} a refined classifications of diverse functions of a model. We will consider modeling as a fundamental instrument of knowledge on processes within complex adaptive systems, as already evoked, and restraining again our question, will focus on \emph{models involving interactions between transportation networks and territories}.
}{
La modélisation joue en Géographie Théorique et Quantitative (TQG) un rôle fondamental. \noun{Cuyala} procède dans~\cite{cuyala2014analyse} à une analyse spatio-temporelle du mouvement de la Géographie Théorique et Quantitative en langue française et souligne l'émergence de la discipline comme une combinaison d'analyses quantitatives (e.g. analyse spatiale et pratiques de modélisation et de simulation) et de construction théoriques. \comment{(Florent) cela remonte à quand ? appliqué à quels champs ?}
L'intégration de ces deux composantes permet la construction de théories à partir de faits stylisés empiriques, qui produisent à leur tour des hypothèses théoriques pouvant être testées sur les données empiriques. Cette approche est née sous l'influence de la \emph{New Geography} dans les pays Anglo-saxons et en Suède. Une histoire étendue de la genèse des modèles de simulation en géographie est faite par \noun{Rey} dans~\cite{rey2015plateforme} avec une attention particulière pour la notion de validation de modèles. L'utilisation de ressources de calcul pour la simulation de modèles est antérieur à l'introduction des paradigmes de la complexité, remontant à \noun{H{\"a}gerstrand}\comment{(Florent) AB, conceptuel, pas computationnel} \comment{(Arnaud) Hagerstrand NON}
 et \noun{Forrester}, \comment{(Arnaud) Forrester $\neq$ géographe}
  pionniers des modèles d'économie spatiale inspirés par la cybernétique. Avec l'augmentation des potentialités de calcul, des transformations épistémologiques ont également suivi, avec l'apparition de models explicatifs comme outils expérimentaux. \noun{Rey} compare le dynamisme des années soixante-dix quand les centres de calcul furent ouverts aux géographes à la démocratisation actuelle du Calcul Haute Performance (calcul sur grille à l'utilisation transparente, voir~\cite{schmitt2014half} pour un exemple des possibilités offertes en terme de calibration et de validation de modèle, réduisant le temps de calcul nécessaire de 30 ans à une semaine - ces techniques jouent un rôle clé pour les résultats que nous obtiendrons par la suite), qui est également accompagnée par une évolution des pratiques~\cite{banos2013pour} et techniques~\cite{10.1371/journal.pone.0138212} de modélisation. La modélisation, et en particulier les modèles de simulation, est vue par beaucoup comme une brique fondamentale de la connaissance : \cite{livet2010} rappelle la combinaison des domaines empirique, conceptuel (théorique) et de la modélisation, avec des retroactions constructives entre chaque. Une modèle peut être un outil d'exploration pour tester des hypothèses, un outil empirique pour valider une théorie sur des jeux de données, un outil explicatif pour révéler des causalités et ainsi des processus internes au système, un outil constructif pour construire itérativement une théorie conjointement avec celle des modèles associés. Ce sont des exemples de fonctions parmi d'autres : Varenne donne dans~\cite{varenne2010simulations} une classification raffinée des diverses fonctions d'un modèle. Nous considérons la modélisation comme un instrument fondamental de connaissance des processus au sein de systèmes complexes adaptatifs, et précisons encore notre question de recherche, qui s'intéressera aux \emph{modèles impliquant des interactions réseaux et territoires}.
}




%%%%%%%%%%%%%%%%%%%%%%%%%%%
\subsection{Modeling Territories and Networks}{Modéliser les territoires et réseaux}

% here overview of different approaches
% Q : do it here, not during quant epistemo part ?


\bpar{
Concerning our precise question of interactions between transportation networks and territories, we propose an overview of existing approaches. Following~\cite{bretagnolle2002time}, the ``\textit{thoughts of specialists in planning aimed to give definitions of city systems, since 1830, are closely linked to the historical transformations of communication networks}''. It is not far from an reversed self-realizing prophecy, in the sense that it is already realized before happening. It implies that ontologies and corresponding models addressed by geographers and planners are closely linked to their current historical preoccupations, thus necessarily limited in scope and purpose. In a perspectivist vision of science~\cite{giere2010scientific} such boundaries are the essence of the scientific entreprise, and as we will argue in chapter~\ref{ch:theory} their combination and coupling in the case of models is a source of knowledge.
}{
Au sujet de notre question précise des interactions entre réseaux de transport et territoires, nous proposons un aperçu des différentes approches. Selon~\cite{bretagnolle2002time}, ``\textit{les idées des spécialistes de la planification cherchant à donner des définitions des systèmes de ville, depuis 1830, sont étroitement liées aux transformations des réseaux de communication}''. \comment{(Florent) la question de la définition de la ville mérite une place plus grande}
 C'est en quelque sorte la prophétie auto-réalisatrice inversée, au sens où elle est déjà réalisée avant d'être formulée. Cela implique que les ontologies et les modèles correspondants proposés par les géographes et les planificateurs sont fortement liés aux préoccupations historiques courantes, ainsi forcément limités en portée et raisons. Dans une vision perspectiviste de la science~\cite{giere2010scientific} de telles limites sont l'essence de l'entreprise scientifique, et comme nous démontrerons en chapitre~\ref{ch:theory} leur combinaison et couplage dans le cas de modèles est une source de connaissance.
}



\subsubsection{Land-Use Transportation Interaction Models}{Modèles LUTI}



\bpar{
A subsequent bunch of literature in modeling interaction between networks and territories can be found in the field of planning, with the so-called \emph{Land-use Transportation Interaction Models}. These works are difficult to be precisely bounded as they may be influenced by various disciplines. For example, from the point of view of Urban Economics, propositions for integrated models have existed for a relatively long term~\cite{putman1975urban}. The variety of possible models has lead to operational comparisons~\cite{paulley1991overview,wegener1991one}. More recently, the respective advantages of static and dynamic modeling was investigated in~\cite{kryvobokov2013comparison}. Generally these type of models operate at relatively small temporal and spatial scales. \cite{wegener2004land} reviewed state of the art in empirical and modeling studies on interactions between land-use and transportation. It is positioned in economic, planning and sociological theoretical contexts, and is relatively far from our geographical approach aiming to also understand long-time processes. Seventeen models are compared and classified, none of which implements actually network endogenous evolution on the relatively small time scales of simulation. A complementary review done in \cite{chang2006models} broadens the scope with inclusion of more general classes of models, such as spatial interaction models (including traffic assignment and four steps models), operational research planning models (optimal localisations), micro-based random utility models, and urban market models. These techniques operate also at small scales and consider at most land-use evolution. \cite{iacono2008models} covers a similar scope with a further emphasis on cellular automata models of land-use change and agent-based models. These type of models are still largely developed and used today, as for example \cite{delons:hal-00319087} which is used for Parisian metropolitan region. The short-term range of application and their operational character makes them useful for planning, what is far from our preoccupation to obtain explicative models for geographical processes. 
}{
Un partie importante de la littérature proposant des modélisations des interactions entre réseaux et territoires se trouve dans le domaine de la planification urbaine, avec les \emph{modèles d'interaction entre usage du sol et transport} (\emph{LUTI}). Ces travaux peuvent être difficiles à cerner car liés à différentes disciplines. Par exemple, du point de vue de l'Economie Urbaine, les propositions de modèle intégrés existent depuis un certain temps~\cite{putman1975urban}. La variété des modèles existants a conduit à des comparaisons opérationnelles~\cite{paulley1991overview,wegener1991one}. Plus récemment, les avantages respectifs des approches statiques et dynamiques a été étudié par~\cite{kryvobokov2013comparison}. \comment{(Florent) ok mais spécifie des durées et échelles d'espace} 
Dans tous les cas, ce type de modèle opère généralement à des échelles temporelles et spatiales relativement faibles.  \cite{wegener2004land} donne un état de l'art des études empiriques et de modélisation sur ce type d'approche des interactions entre usage du sol et transport. Le positionnement théorique est plutôt proche des disciplines de l'Economie, de la Planification et de la Sociologie, et relativement de nos raisonnements géographiques qui se veulent de comprendre également des processus sur le temps long.\comment{(Florent) d'abord dresser le tableau des disciplines qui s'y intéressent, pourquoi et comment}
 Pas moins de dix-sept modèles sont comparés et classifiés, parmi lesquels aucun n'inclut une évolution endogène du réseau de transport sur les échelles de temps relativement petites des simulations. Une revue complémentaire est faite par~\cite{chang2006models}, élargissant le contexte avec l'inclusion de classes plus générales de modèles, comme des modèles d'interactions spatiales (parmi lesquels l'attribution du traffic et les modèles à quatre temps), les modèles de planification basés sur la recherche opérationnelle (optimisation des localisations), les modèles microscopiques d'utilité aléatoire, et les modèles de marché foncier. Toutes ces techniques opèrent également à une petite échelle et considèrent au plus l'évolution de l'usage du sol. \cite{iacono2008models} couvre un horizon similaire avec une emphase supplémentaire sur les modèles à automates cellulaires d'évolution d'usage du sol et les modèles basés agent. Les modèles LUTI sont toujours largement étudiés et appliqués, comme par exemple \cite{delons:hal-00319087} qui est utilisé pour la région métropolitaine parisienne. La courte portée temporelle d'application de ces modèles et leur nature opérationnelle les rend utiles pour la planification, \comment{(Florent) détailler ce que cela veut dire aidera certainement à mieux positionner par rapport au planning}
 ce qui est assez loin de notre souci d'obtenir des modèles explicatifs de processus géographiques.
}




\subsubsection{Network Growth}{Croissance du Réseau}

% economic models
%\cite{yerra2005emergence} % : cost-driven model of nw reinforcement (// slime mould)
%\cite{louf2013emergence} % : trade-off between cost and benefits due to flows -> compare with lutecia rules ?
%\cite{xie2009modeling} review of network growth economic appraoches
%\cite{bigotte2010integrated} % : planning network - similar to nw growth ? - hierarchy in cities and nw
 
 % geometric - local optimization
%\cite{barthelemy2008modeling}
%\cite{courtat2011mathematics} % measures and morphogenesis. Vision of morphogenesis as living organism : nuance that in theory.
%\cite{de2007netlogo} % geom rules in Tijuana model
%\cite{rui2013exploring} % local based optimisation morphogenesis model. RQ : quote only physicists work -> justification for extended quantitative epistemology ?
%\cite{yamins2003growing} strange model
 
% biological nws

%\cite{tero2010rules} % physarum : biological nw heuristics
%\cite{tero2006physarum} % potentialities of physarum machines, here for routing.
%\cite{adamatzky2010road} % planning absurdities
%\cite{zhu2013amoeba} % TSP solving : long range correlations.


\bpar{
Network growth can be used to design modeling entreprises that aim to endogenously explain growth of transportation networks, generally from a bottom-up point of view, i.e. by exhibiting local rules that would allow to reproduce network growth over long time scales (generally the road network). Economists have proposed such models: \cite{zhang2007economics} reviews transportation economics literature on network growth within an endogenous growth theory~\cite{aghion1998endogenous}, recalling the three main features studied by economists on that subject that are road pricing, infrastructure investment and ownership regime, and describes an analytical model combining the three.
\cite{xie2009modeling} develops a broad review on network growth modeling extending to other fields: transportation geography early developed empirical-based models but which did concentrate on topology reproduction rather than on mechanisms according to~\cite{xie2009modeling}; statistical models on case studies provide mitigated conclusions on causal relations between offer and demand; economists have studied infrastructure provision from both microscopic and macroscopic point of views, generally non-spatial; network science has provided toy-models of network growth based on structural and topological rules rather on rules inspired from real processes. An other approach not mentioned that we will develop further is biologically inspired network design. We first give some example of economic-based and geometrical-based network growth modeling attempts. \cite{yerra2005emergence} shows through a reinforcement economic model including investment rule based on traffic assignment that local rules are enough to make hierarchy of roads emerge for a fixed land-use. A very similar model in~\cite{louf2013emergence} with simpler cost-benefits obtains the same conclusion. Whereas these models based on processes focus on reproducing macroscopic patterns of networks (typically scaling), geometrical optimization models aim to ressemble topologically real networks. \cite{barthelemy2008modeling} proposes a model based on local energy optimization but it stays very abstract and unvalidated. The morphogenesis model given in~\cite{courtat2011mathematics} using local potential and connectivity rules, even if not calibrated, seems to reproduce more reasonably real street patterns. Very close work is done in~\cite{rui2013exploring}.
Other tentatives \cite{de2007netlogo,yamins2003growing} are closer to procedural modeling~\cite{lechner2004procedural,watson2008procedural} and therefore not of interest in our purpose as they can difficultly be used as explicative models. Finally, an interesting and original approach to network growth are biological networks. These belong to the field of morphogenetic engineering pioneered by \noun{Doursat} that aim to design artificial complex system inspired from natural complex systems and in which a control of emerging properties is possible~\cite{doursat2012morphogenetic}. \emph{Physarum Machines}, that are models of a self-organized mould (slime mould) have been shown to provide efficient bottom-up solution to computationally heavy problems such as routing problems~\cite{tero2006physarum} or NP-complete navigation problems such as the Travelling Salesman Problem~\cite{zhu2013amoeba}. It has been shown to produce networks with Pareto-efficient cost-robustness properties~\cite{tero2010rules}, relatively close in shape to real networks (under certain conditions, see~\cite{adamatzky2010road}). This type of models can be of interest for us since auto-reinforcement mechanisms based on flows are analog to mechanisms of link reinforcement in transportation economics.
}{
La croissance de réseaux est pratiquée dans des entreprises de modélisation qui cherchent à expliquer de manière endogène \comment{(Florent) de quel point de vue ?}
la croissance des réseaux de transport, généralement d'un point de vue \emph{bottom-up}, i.e. en mettant en évidence des règles locales qui permettraient de reproduire la croissance du réseau sur de longues échelles de temps (souvent le réseau de rues). Les économistes ont proposés des modèles de ce type : \cite{zhang2007economics} passe en revue la littérature en économie de transports sur la croissance des réseaux dans le contexte d'une théorie endogène de la croissance~\cite{aghion1998endogenous}, rappelant les trois aspects principalement traités par les économistes sur le sujet, qui sont la tarification routière, l'investissement en infrastructures et le régime de propriété, et propose finalement un modèle analytique combinant les trois.
\cite{xie2009modeling} propose une revue étendue de la modélisation de croissance des réseaux, en prenant en compte d'autres champs : la géographie des transports a développé très tôt des modèles basés sur des faits empiriques mais qui se sont concentrés sur reproduire la topologie plutôt que sur les mécanismes selon~\cite{xie2009modeling} ; les modèles statistiques sur des cas d'étude fournissent des conclusions très mitigées sur les relations causales entre offre et demande \comment{(Florent) du coup ce n'est pas que pur réseau a priori}[todo : define what we mean by network]
; les économistes ont étudié la production d'infrastructure à la fois d'un point de vue microscopique et macroscopique, généralement non spatiaux ; la science des réseaux a produit des modèles jouet de croissance de réseau qui se basent sur des règles topologiques et structurelles plutôt que des règles se reposant sur des processus inspirés de faits réels. Une autre approche qui n'est pas mentionnée et que nous allons approfondir est la conception de réseau inspirée de la biologie. Nous donnons pour commencer des exemples d'études utilisant des concepts économiques ou géométriques pour modéliser la croissance de réseau. \cite{yerra2005emergence} montre avec un modèle économique basé sur des processus auto-renforçants et incluant une règle d'investissement basée sur l'attribution du trafic, que des règles locales sont suffisantes pour faire émerger une hiérarchie du réseau routier à usage du sol fixé. Une modèle très similaire donnée par~\cite{louf2013emergence} avec des fonctions coûts-bénéfices plus simples obtient une conclusion similaire. \comment{(Florent) devrais rentrer plus dans le détail d'un ou deux modèles}
Alors que ces modèles basés sur des processus cherchent à reproduire des motifs macroscopiques des réseaux (typiquement les lois d'échelle), les modèles d'optimisation géométrique cherchent à ressembler à des réseaux réels dans leur topologie. \cite{barthelemy2008modeling} décrit un modèle basé sur une optimisation locale de l'énergie, mais ce modèle reste très abstrait et non validé. Le modèle de morphogenèse de~\cite{courtat2011mathematics} qui utilise des potentiels locaux et des règles de connectivité, même s'il n'est pas calibré, semble reproduire de manière plus raisonnable des motifs réels des réseaux de rues. Un modèle très proche est décrit dans~\cite{rui2013exploring}.
D'autres tentatives comme~\cite{de2007netlogo,yamins2003growing} sont plus proches de la modélisation procédurale~\cite{lechner2004procedural,watson2008procedural} et pour cette raison n'ont pas d'intérêt pour notre cas puisqu'ils peuvent difficilement être utilisés comme modèles explicatifs. \comment[JR]{développer plus pourquoi la modélisation procédurale n'est pas satisfaisante : forme ``fidèle'' en général pas à la bonne échelle ; penser qu'il s'agit de modèles de morphogenèse urbaine est une erreur grossière de POM à la mauvaise échelle. typiquement nos techniques pour générer des données synthétiques en exp mixture et connexification sont de ce type, et pour cela nous ne les explorons pas mais utilisons comme générateur de données uniquement.}
 Enfin, une approche originale et intéressante à la croissance des réseaux sont les réseaux biologiques. Ils appartiennent au champ de l'ingénierie morphogénétique dont \noun{Doursat} est un pionnier, qui vise à concevoir des systèmes complexes artificiels inspirés de systèmes complexes naturels et sur lesquels un contrôle des propriétés émergentes est possible~\cite{doursat2012morphogenetic}. Les \emph{Machines Physarum}, qui sont des modèles d'une moisissure auto-organisée (\emph{slime mould}) ont été prouvés comme résolvant de manière efficiente et par le bas des problèmes computationnellement lourds comme des problème de routage~\cite{tero2006physarum} ou des problèmes de navigation NP-complets comme le Problème du Voyageur de Commerce~\cite{zhu2013amoeba}. \comment{(Florent) cela n'est pas de première importance je pense}
Ils produisent des réseaux ayant des propriétés de coût-robustesse Pareto-efficientes~\cite{tero2010rules}, \comment{(Florent) et alors, est ce que cela correspond à une réalité empirique ? repartir des trois sphères Muller Livet Sanders peut aider (empirique, conceptuel, du modèle)}
relativement proches en forme de réseaux réels (sous certaines conditions, voir~\cite{adamatzky2010road}). Ce type de modèles peut être d'intérêt dans notre cas puisque les processus d'auto-renforcement basés sur les flots sont analogues aux mécanismes de renforcement de lien en économie des transports.
}

\comment{comparison with Francois model for french railway (nothing published yet) - C Mimeur (thèse soutenue ?)}

\comment{\cite{levinson2012forecasting} mécanismes induisant la croissance du réseau, gouvernance et économiques, très détaillés, basé sur enquêtes quali et modèle stats fittés sur vraies données}

\comment{\cite{xie2009jurisdictional} compares centralized vs decentralized network growth}


\comment{\cite{levinson2003induced} fits statistical models, including multinomial logit, to find driver of highway network growth (on Twin Cities). Basic variables (length, change in accessibility) have expected behavior ; there is a difference between interstate and local investments : local road growth is not affected by cost. Corresponds to requirement of equity in local territorial accessibility ?}

\comment{\cite{chen2006effectiveness} : la simulation comme outil pour apprendre aux élèves ingénieurs. Intéressant à utiliser pour l'aspect performatif, feedback des modèles sur les situations réelles / illustration des différents objectifs de chaque domaine : pourquoi et comment c'est intéressant de prendre en compte certains aspects selon les objectifs / perspectivisme appliqué : faire ce projet , l'évoquer ici.}

\comment{\cite{mimeur:tel-01451164} la thèse de Mimeur est un pont intéressant entre géographie et approches éco de Levinson (modèle de croissance type slime mould ?). plus fait des stats spatiales pour lier croissance pop et accessibilité : checker si même résulats quand fera spatio-temp causalities sur réseau ferré et autoroutier et croissance pop. remarque : trucs bizzares, essaie d'expliquer pour petites villes, mais pas approprié, pb du choix de l'échelle, de ce qui est du bruit et du signal - semble tout mélanger : importance du preprocessing et traitement du signal (cf correlations des taux de croissance). Tester effets fixes régions/départements ? fait GWR finalement ?}




%%%%%%%%%%%%%%%%%%
\subsection{Modeling co-evolution}{Modéliser la co-évolution}


\subsubsection{Hybrid Modeling}{Modélisation Hybride}



%\cite{bigotte2010integrated}
%\cite{levinson2005paving} % markov chain : not really modeling but more statistics.
%\cite{raimbault2014hybrid}



\bpar{
Models of simulation implementing a coupled dynamic between urban growth and transportation network growth are relatively rare, and always rather poor from a theoretical and thematic point of view. A generalization of the geometrical local optimization model described before was developed in~\cite{barthelemy2009co}. % pb of scales, def of coevolution, thematic meaning of assumptions, etc.
As for the road growth model of which it is an extension, no thematic nor theoretical justification of local mechanisms is provided, and the model is furthermore not explored and no geographical knowledge can be drawn from it. \cite{levinson2007co} adopts a more interesting economic approach, similar to a four step model (gravity-based origin-destination flows generation, stochastic user equilibrium traffic assignment) including travel cost and congestion, coupled with a road investment module simulating toll revenues for constructing agents, and a land-use evolution module updating actives and employments through discrete choice modeling. The experiments showed that co-evolving network and land uses lead to positive feedbacks reinforcing hierarchy, but are far from satisfying for two reasons: first network topology does not really evolve as only capacities and flows change within the network, what means that more complex mechanisms on longer time scales are not taken into account, and secondly the conclusions are very limited as model behavior is not known since sensitivity analysis is done on few one-dimensional spaces: exhaustive mechanisms stay thus unrevealed as only particular cases are described in the sensitivity analysis. From an other point of view, \cite{levinson2005paving} is also presented as a model of co-evolution, but corresponds more to coupled statistical analysis as it relies on a Markov-chain predictive model. \cite{rui2011urban} gives a model in which coupling between land-use and network growth is done in a weak paradigm, land-use and accessibility having no feedback on network topology evolution. \cite{achibet2014model} describes a co-evolution model at a very small scale (scale of the building), in which evolution of both network and buildings are ruled by a same agent (influenced differently by network topology and population density) what implies a too strong simplification of underlying processes. Finally, a simple hybrid model explored and applied to a toy planning example in~\cite{raimbault2014hybrid}, relies on urban activities accessibility mechanisms for settlement growth with a network adapting to urban shape. The rules for network growth are too simple to capture processes we are interested in, but the model produces at a small scale a broad range of urban shapes reproducing typical patterns of human settlements.
}{
Les modèles de simulation qui incluent un couplage des dynamiques de la croissance urbaine et du réseau de transport sont relativement rares, et pour la plupart au stade de modèles stylisés. Une généralisation du modèle d'optimisation locale géométrique décrit précédemment a été développé dans~\cite{barthelemy2009co}. Comme pour le modèle de croissance de réseau routier dont il est l'extension, les mécanismes locaux n'ont pas de justification théorique ou thématique, et le modèle n'est de plus pas exploré et aucune connaissance géographique ne peut en être tirée. \cite{levinson2007co} prend une approche économique plus intéressante du point de vue des processus de développement de réseau impliqués, similaire à un modèle à quatre étapes (génération de flux origine-destination basés sur la gravité, attribution du traffic par Equilibre Utilisateur Stochastique) qui inclut coût de transport et congestion, couplé avec un module d'investissement routier qui simule les revenus des péages pour les agents qui construisent, et un module d'évolution d'usage du sol qui met à jour les actifs et emplois par modélisation de choix discrets. Les expériences montrent que l'usage du sol et le réseau en co-évolution mène à des retroactions positives renforçant les hiérarchies, mais sont loin d'être satisfaisantes pour deux raisons : d'une part la topologie du réseau n'évolue pas à proprement parler puisque seules les capacités et les flux changent dans le réseau, ce qui signifie que des mécanismes plus complexes sur de plus longues échelles de temps ne sont pas pris en compte, et d'autre part les conclusions sont assez limitées puisque le comportement du modèle n'est pas connu, les analyses de sensibilité étant faites sur un petit nombre d'espaces unidimensionnels : les mécanismes exhaustifs restent ainsi inconnus comme seuls des cas particuliers sont donnés dans l'analyse de sensibilité. D'un autre point de vue, \cite{levinson2005paving} est aussi présenté comme un modèle de co-évolution mais correspond plus à une analyse statistique couplée puisqu'elle repose sur un modèle prédictif à chaîne de Markov. \cite{rui2011urban} décrit un modèle dans lequel le couplage entre usage du sol et la topologie du réseau est fait par un paradigme faible, l'usage du sol et l'accessibilité n'ayant pas de retroaction sur la topologie du réseau. \cite{achibet2014model} décrit un modèle de co-évolution à une très petite échelle (échelle du bâtiment), dans lequel l'évolution du réseau et des bâtiments sont tous les deux régis par un agent commun (qui est influencé différemment par la topologie du réseau et la densité de population) ce qui implique une simplification trop grande des processus sous-jacents. Enfin, un modèle hybride simple exploré et appliqué à un exemple jouet de planification dans~\cite{raimbault2014hybrid}, repose sur les mécanismes d'accès aux activités urbaines pour la croissance des établissements avec un réseau s'adaptant à la forme urbaine. Les règles pour la croissance du réseau sont trop simples pour capturer les processus qui nous intéressent, mais le modèle produit à une petite échelle une large gamme de formes urbaines qui reproduisent les motifs typiques des établissements humains. A une échelle macroscopique et plus proche de la modélisation de système urbains que nous développerons dans la section suivante, \cite{baptiste1999interactions} propose de coupler le modèle de croissance urbaine basé sur les migrations (introduit par l'application de la synergétique au système de ville par \noun{Sanders} dans~\cite{sanders1992systeme}) avec un mécanisme d'auto-renforcement pour le réseau routier sans modification topologique (retroaction positive par seuils du différentiel flux-capacité sur la capacité). Guère de conclusions générales ne peuvent cependant être tirées de ce travail, outre que ce couplage permet de faire émerger une configuration hiérarchique (mais on sait par ailleurs que des modèles plus simples, un attachement préférentiel uniquement par exemple, permettent de reproduire ce fait stylisé) et que l'ajout du réseau produit un espace moins hiérarchique, permettant à des villes moyennes de bénéficier de la rétroaction du réseau de transport.
}

\comment{(Florent) pas assez de prise de hauteur sur cette partie pour une fois, on ne voit pas le tableau d'ensemble}

\comment{(Florent) : cf renvoi remarques generales sur chapitre 1}

\comment{(Juste) \cite{baptistemodeling} : paper, quite the same as in thesis by Baptiste.
 \cite{badariotti2007conception,moreno2012automate} : remus and raumulus, inspiration for rbd model. relire thèse Moreno.
}

\comment[JR]{\cite{blumenfeld2010network} : hybrid model (largely discussed by Clara) ; network growth induces migration ; would be interesting to test its abilities to produce various causality regimes (note : may be one indicator of how a model captures co-evolution ?)}




\subsubsection{Urban Systems Modeling}{Modélisation de Systèmes Urbains}



\bpar{
An approach rather close to our current questioning is the one of integrated modeling of system of cities. In the continuity of Simpop models for city systems modeling, \noun{Schmitt} described in~\cite{schmitt2014modelisation} the SimpopNet model which aim was precisely to integrate co-evolution processes in system of cities on long time scales, typically via rules for hierarchical network development as a function of cities dynamics coupled with these that depends on network topology. Unfortunately the model was not explored nor further studied, and furthermore stayed at a toy-level. \noun{Cottineau} proposed transportation network endogenous growth as the last building bricks of her Marius productions but it stayed at a conceptual construction stage. We shall position more in that stream of research in this thesis.
}{
Une approche relativement proche des précédentes, mais ayant des caractéristiques propres, est celle de la modélisation intégrée des systèmes de villes. Dans la continuité des modèles Simpop pour modéliser les systèmes de villes, \noun{Schmitt} décrit dans~\cite{schmitt2014modelisation} le modèle SimpopNet qui vise à précisément intégrer les processus de co-évolution dans les systèmes de villes à longue échelle temporelle, typiquement par des règles pour un développement hiérarchique du réseau comme fonction des dynamiques des villes, couplées à celles-ci qui dépendent de la topologie du réseau. Malheureusement le modèle n'a pas été exploré ni étudié de manière plus approfondie, et de plus est resté au niveau de modèle jouet. \noun{Cottineau} propose une croissance endogène des réseaux de transport comme la dernière brique de construction de ses productions Marius~\cite{cottineau2014evolution} mais cela reste à un niveau conceptuel puisque cette brique n'a pas encore été spécifiée ni implémentée. Il n'existe à notre connaissance pas de modèle empirique ou appliqué à un cas concret se basant sur une approche de la co-évolution par les systèmes urbains vus par la Théorie Evolutive des Villes. Nous nous positionnerons particulièrement dans cette lignée de recherche dans cette thèse, vu l'importance que prendra la Théorie Evolutive dans notre démarche Théorique et de Modélisation comme nous le détaillerons par la suite. L'ensemble des briques est nécessaire pour comprendre les implications de ce positionnement, mais le lecteur pressé pourra directement consulter le chapitre~\ref{ch:theory} pour une synthèse des implications théoriques à différents niveaux d'abstraction. Typiquement, les hypothèses épistémologiques fondamentales tel le rôle des relations et de la configuration spatiales, ou la présence d'un équilibre - nous considérons les systèmes urbains comme des systèmes complexes adaptatifs, auto-organisés loin de l'équilibre, sont typiques de cette approche si on les considère conjointement. On voit bien l'opposition aux principes épistémologiques de l'économie géographique : \cite{fujita1999evolution} introduit par exemple un modèle évolutionnaire capable de reproduire une hiérarchie urbaine et une organisation typique de la Théorie des Places Centrales, mais repose toujours sur la notion d'équilibres successifs, et surtout considère un modèle ``à-la-krugman'' c'est à dire un espace à une dimension homogène. Cette approche peut être instructive sur les processus économiques en eux-mêmes mais aucunement sur les processus géographiques, qui incluent le déroulement des processus économiques dans l'espace géographique dans lequel les particularités sont essentielles. Notre travail s'attellera à montrer dans quelle mesure cette structure de l'espace peut être importante et également explicative, puisque les réseaux , et encore plus les réseaux physiques induisent des processus dépendants au chemin spatio-temporel et donc sensibles au singularités locales et propices aux bifurcations induites par la combinaison de celles-ci et de processus à d'autres échelles (par exemple la centralité induisant un flux).
}




\subsubsection{Co-evolution}{Co-évolution}

\comment{(Florent) constat à livrer d'emblée (pas de modèle de co-evolution dans algoSR)}[détailler ici que en effet par vraiment en notre sens.]


\comment[JR]{une première entrée simple sur la co-evolution : couplage fort pour l'instant - sinon cf théorie (d'ailleurs y revenir sur multiples causality regimes : more links theory - reste) - \cite{paulus2004coevolution} : evidence of co-evolution phenomena -> put when introduce co-evol. ; développer ``les lacunes à combler'' : modèles fortement couplés plus ou moins multi-processes et multi-échelles ? (dans le temps au moins) - ce que nos modèles apportent - dans une vision de théorie intégrative.}





