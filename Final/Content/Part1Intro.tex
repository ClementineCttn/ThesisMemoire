

%\chapter*{Part I Introduction}{Introduction de la Partie I}
\chapter*{Introduction de la Partie I}


% to have header for non-numbered introduction
%\markboth{Introduction}{Introduction}


%\headercit{}{}{}


%---------------------------------------------------------------------------


















%---------------------------------------------------------------------------


\chapter*{Définitions prélimimaires}

Il est nécessaire de fixer pour commencer les définitions de notions qui joueront un rôle clé tout au long de notre raisonnement. Nous adoptons la stratégie suivante : les définitions données sont assez générales pour que les raffinements lorsqu'ils auront lieu précisent ces notions. Une fois qu'une notion aura été raffinée, son utilisation fera référence à l'ensemble de la profondeur (sauf utilisation particulière locale qui sera alors précisée explicitement). Cette stratégie permet d'une part d'alléger la lecture, et d'autre part favorise une lecture non-linéaire, vu que la profondeur complète ne sera pas nécessaire à toute étape pour une compréhension au premier ordre des connaissances construites. Lorsqu'une référence précise n'est pas donnée, les définitions sont inspirées de~\cite{hypergeo}. 


\subsection*{System}{Système}

Un \emph{Système} est composé ``\textit{d'un ensemble d'entités en interaction}''. Différentes formalisations équivalentes


\subsection*{Models}{Modèles et Ontologies}



\subsection*{Cities, System of Cities, Territories}{Villes, Systèmes de Villes, Territoires}




\subsection*{Causality}{Causalité}



\subsection*{Model Coupling}{Couplage de Modèles, Modèles Intégratifs}











