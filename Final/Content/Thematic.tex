

% Chapter 




%\chapter{Interactions between Networks and Territories}{Interactions entre Réseaux et Territoires} % Chapter title
\chapter{Interactions entre Réseaux et Territoires}


\label{ch:thematic} % For referencing the chapter elsewhere, use \autoref{ch:name} 


%%  Thematic chapter framing geographically the subject.
%%   and reviewing state of the art
%%   and why modeling : evolutive theory of urban systems etc ; multimodeling simfamily etc
%%  
%%   Q  : example to introduce theory ?
%
%   Modelography.  (non-exhaustive) : classification according to purpose, theme, scale, etc.
%   Why dynamic models of ``co-evolution''  ?
%   definition of terms, contextualisation, etc.  (le what/where d'Arnaud ; ontology de Anne)



%----------------------------------------------------------------------------------------

%\headercit{If you are embarrassed by the precedence of the chicken by the egg or of the egg by the chicken, it is because you are assuming that animals have always be the way they are}{Denis Diderot}{\cite{diderot1965entretien}}

\headercit{Si la question de la priorit{\'e} de l'\oe{}uf sur la poule ou de la poule sur l'\oe{}uf vous embarrasse, c'est que vous supposez que les animaux ont {\'e}t{\'e} originairement ce qu'ils sont {\`a} pr{\'e}sent.%\comment{(Florent) génial cette citation}
}{Denis Diderot}{\cite{diderot1965entretien}}
%\headercit{Si la question de la priorit{\'e} de l'\oe{}uf sur la poule ou de la poule sur l'\oe{}uf vous embarrasse, c'est que vous supposez que les animaux ont {\'e}t{\'e} originairement ce qu'ils sont {\`a} pr{\'e}sent.}{Denis Diderot}{\cite{diderot1965entretien}}


\bigskip



\bpar{
This analogy is ideal to evoke the questions of causality and processes in territorial systems. When trying to tackle naively our preliminary question, some observers have qualified the identification of causalities in complex systems as ``chicken and egg'' problems : if one effect appears to cause another and reciprocally, how can one disentangle effective processes ? This vision is often present in reductionist approaches that do not postulate an intrinsic complexity in studied systems. The idea that Diderot suggests is the notion of \emph{co-evolution} that is a central phenomenon in evolutive dynamics of Complex Adaptive Systems as \noun{Holland} develops in~\cite{holland2012signals}. He links the notion of emergence (that is ignored in a reductionist vision), in particular the emergence of structures at an upper scales from the interactions between agents at a given scale, materialized generally by boundaries, that become crucial in the coevolution of agents at any scales : the emergence of one structure will be simultaneous with one other, each exploiting their interrelations and generated environments conditioned by their boundaries. We shall explore these ideas in the case of territorial systems in the following.
}{
Cette analogie est idéale pour introduire les notions de causalité et de processus dans les systèmes territoriaux. En voulant traiter naïvement des questions similaires à notre question de recherche préliminaire, certains on qualifiés les causalités au sein de systèmes complexes comme un problème ``de poule et {\oe}uf'' \comment{(Florent) parler à ce stade de la constroverse Offner 93} \comment{(Arnaud) :) }
 : si un effet semble causer l'autre et réciproquement, comment est-il possible d'isoler les processus correspondants ? Cette vision est souvent présente dans les approches réductionnistes qui ne postulent pas une complexité intrinsèque au sein des systèmes étudiés. L'idée suggérée par \noun{Diderot} est celle de \emph{co-evolution} qui est un phénomène central dans les dynamiques évolutionnaires des Systèmes Complexes Adaptatifs comme \noun{Holland} élabore dans~\cite{holland2012signals}. Il fait le lien entre la notion d'émergence (ignorée dans les approches réductionnistes)\comment{(Florent)la encore très epistemo, renforcer connaissance empirique de ces interactions particulieres et en faire état ici}
 , en particulier l'émergence de structures à une plus grand échelle par les interactions entre agents à une échelle donnée, en général concrétisée par un systèmes de limites, qui devient cruciale pour la co-évolution des agents à toutes les échelles : l'émergence d'une structure sera simultanée avec une autre, chacune exploitant leur interrelations et environnements générés conditionnés par le système de limites. Nous explorerons ces idées pour le cas des systèmes territoriaux par la suite.
 \comment{(Florent)c'est seulement là que tu dis que les syst. territoriaux sont une déclinaison des questionnement précédents}
}

\bpar{
This introductive chapter aims to set up the thematic scene, the geographical context in which further developments will root. It is not supposed to be understood as an exhaustive literature review nor the fundamental theoretical basement of our work (the first will be an object of chapter~\ref{ch:quantepistemo} whereas the second will be earlier tackled in chapter~\ref{ch:theory}), but more as narration aimed to introduce typical objects
 and views and construct naturally research questions.
}{
Ce chapitre introductif est destiné à poser le cadre thématique, le contexte géographique sur lesquels les développements suivants se baseront. Il n'est pas supposé être compris comme une revue de littérature exhaustive ni comme les fondations théoriques fondamentales de notre travail (le premier point étant l'objet du chapitre~\ref{ch:quantepistemo} tandis que le second sera traité plus tôt dans le chapitre~\ref{ch:theory}), mais plutôt comme une construction narrative ayant pour but d'introduire nos objets et positions d'étude,\comment{(Florent) le faire plutot que l'annoncer}
 afin de construire naturellement des questions de recherche précises.
}



\stars


\textit{Ce chapitre est entièrement inédit.}







%-------------------------------



























