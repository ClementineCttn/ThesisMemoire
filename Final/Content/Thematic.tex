

% Chapter 




%\chapter{Interactions between Networks and Territories}{Interactions entre Réseaux et Territoires} % Chapter title
\chapter{Interactions entre Réseaux et Territoires}


\label{ch:thematic} % For referencing the chapter elsewhere, use \autoref{ch:name} 




%----------------------------------------------------------------------------------------

%\headercit{If you are embarrassed by the precedence of the chicken by the egg or of the egg by the chicken, it is because you are assuming that animals have always be the way they are}{Denis Diderot}{\cite{diderot1965entretien}}

%\headercit{Si la question de la priorit{\'e} de l'\oe{}uf sur la poule ou de la poule sur l'\oe{}uf vous embarrasse, c'est que vous supposez que les animaux ont {\'e}t{\'e} originairement ce qu'ils sont {\`a} pr{\'e}sent.
%}{Denis Diderot}{\cite{diderot1965entretien}}


\bigskip


\bpar{
This analogy is ideal to evoke the questions of causality and processes in territorial systems. When trying to tackle naively our preliminary question, some observers have qualified the identification of causalities in complex systems as ``chicken and egg'' problems : if one effect appears to cause another and reciprocally, how can one disentangle effective processes ? This vision is often present in reductionist approaches that do not postulate an intrinsic complexity in studied systems. The idea that Diderot suggests is the notion of \emph{co-evolution} that is a central phenomenon in evolutive dynamics of Complex Adaptive Systems as \noun{Holland} develops in~\cite{holland2012signals}. He links the notion of emergence (that is ignored in a reductionist vision), in particular the emergence of structures at an upper scales from the interactions between agents at a given scale, materialized generally by boundaries, that become crucial in the coevolution of agents at any scales : the emergence of one structure will be simultaneous with one other, each exploiting their interrelations and generated environments conditioned by their boundaries. We shall explore these ideas in the case of territorial systems in the following.
}{
Pour mieux visualiser les notions de causalités circulaires dans les systèmes complexes, et pourquoi celles-ci peuvent conduire à des paradoxes en apparence, l'image fournie par \noun{Diderot} dans~\cite{diderot1965entretien} est éclairante : ``\textit{Si la question de la priorit{\'e} de l'\oe{}uf sur la poule ou de la poule sur l'\oe{}uf vous embarrasse, c'est que vous supposez que les animaux ont {\'e}t{\'e} originairement ce qu'ils sont {\`a} pr{\'e}sent}''. En voulant traiter naïvement des questions similaires induites par notre problématique introduite précédemment, les causalités au sein de systèmes complexes géographiques peuvent être présentées comme un problème ``de poule et {\oe}uf'' : si un effet semble causer l'autre et réciproquement, est-il possible et même pertinent de vouloir isoler les processus correspondants ? Cette question est bien connue des planificateurs des transports, comme le rappelle la notion des ``effets structurants'' qui fait débat depuis un certain temps au moins dans la communauté scientifique \comment[FL]{non cela va trop vite}. Une vision simplifiée, selon laquelle on peut attribuer des rôles systématiques à une composante particulière, est souvent présente dans les approches réductionnistes qui ne postulent pas une complexité intrinsèque au sein des systèmes étudiés.\comment[FL]{mots pas clairs ; a la place : amener co-évolution} L'idée suggérée par \noun{Diderot} est celle de \emph{co-évolution} qui est un phénomène central dans les dynamiques évolutionnaires des Systèmes Complexes Adaptatifs. \cite{holland2012signals} en propose une théorie. Il fait le lien entre l'émergence de structures à une échelle supérieure avec les interactions entre agents à une échelle donnée, en général concrétisée par un systèmes de limites\comment[FL]{pas clair}, qui devient cruciale pour la co-évolution des agents à toutes les échelles : l'émergence d'une structure sera simultanée avec une autre, chacune exploitant leurs interrelations et environnements générés conditionnés par le système de limites.\comment[FL]{rupture : le fil n'est pas clair} Nous explorerons ces idées pour le cas des systèmes territoriaux par la suite. Ceux-ci illustrent parfaitement ces problématiques, et sont typiques de systèmes dans lesquels cette complexité\comment[FL]{laquelle} est cruciale pour une appréhension raisonnable des mécanismes impliqués dans leurs dynamiques. Un certain nombre d'illustrations concrètes\comment[FL]{de quoi ?} seront d'abord proposées pour formuler nos questionnements dans des contextes géographiques donnés.\comment[FL]{phrase inutile}
}



\bpar{
This introductive chapter aims to set up the thematic scene, the geographical context in which further developments will root. It is not supposed to be understood as an exhaustive literature review nor the fundamental theoretical basement of our work (the first will be an object of chapter~\ref{ch:quantepistemo} whereas the second will be earlier tackled in chapter~\ref{ch:theory}), but more as narration aimed to introduce typical objects and views and construct naturally research questions.
}{
Ce chapitre introductif est destiné à poser le cadre thématique, les contextes géographiques sur lesquels les développements suivants se baseront.\comment[FL]{plus haut} Il n'est pas supposé être compris comme une revue de littérature exhaustive ni comme les fondations théoriques fondamentales de notre travail, le premier point étant l'objet du chapitre~\ref{ch:modelinginteractions} tandis que le second sera traité systématiquement dans le chapitre~\ref{ch:theory} lorsque le recul nécessaire aura été progressivement construit. Il doit plutôt être lu comme une construction narrative ayant pour but d'introduire nos objets et positions d'étude.\comment[FL]{cela sera dans l'intro generale} La notion de co-évolution est particulièrement pertinente\comment[FL]{oui : a remonter} pour comprendre les interactions entre territoires et réseaux. Dans une première section~\ref{sec:networkterritories}, nous préciserons l'approche prise de l'objet territoire, et dans quelle mesure celui-ci naturellement implique la considération des réseaux de transport pour la compréhension des dynamiques couplées. Ces considérations abstraites seront illustrées par des cas d'étude concrets dans la deuxième section~\ref{sec:casestudies}, choisis très différents pour comprendre les enjeux d'universalité sous-jacents. Enfin, dans la troisième section~\ref{sec:qualitative},des éléments d'observation de terrain effectués en Chine préciseront encore ces exemples aux échelles microscopique et mesoscopique. \comment[FL]{reprendre}
}



\stars


\textit{Ce chapitre est entièrement inédit.}







%-------------------------------



























