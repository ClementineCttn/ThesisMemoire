



%----------------------------------------------------------------------------------------

\newpage

%\section[A Geographical Theory][Une Théorie Géographique]{A Geographical Theory for Networks and Territories}{Une Théorie Géographique des Territoires et des Réseaux}

\section{A Geographical Theory}{Une Théorie Géographique}

\label{sec:theory}

%----------------------------------------------------------------------------------------


%%%%%%%
%% -- Points restant à éclaircir --

% - \comment{(Florent)  peut être que des schémas pourraient aider le lecteur} pour la transition d'échelle morphogénétique notamment
% - \comment{(Florent) et y'a t'il des territorial niches ? par ailleurs pourquoi pas tester hypothèse niche mais pourquoi pas autre chose ?} -> a creuser selon résultats empiriques et de modélisation
% - sur morphogenese : \cite{desmarais1992premisses} ; \cite{levy2005formes}
% - \comment{(Arnaud) Notion de Niche}
%  - definition of scale and stationarity
%        \textit{Equivalence between existence of discrete scales and discrete stationarity levels ?}
%         // equivalence between space and time - ergodicité / non ergodicité
%  - Sur la demo stationarité locale : \comment{(Florent) on ne peut pas te suivre, on se demande si par ailleurs on doit le faire} ; \comment{(Florent) manque toute une discussion sur els objets géographiques, villes/systèmes de villes, etc. et les échelles de temps des dynamiques territoriales}
%    !!! Plus : !!! on suppose que la décomposition modulaire est fixe dans le temps, voir avec une décomposition variable ?
% \comment{(Florent) je ne comprends pas du tout sur quoi tu bâtis à ce stade. les objectifs, je pensais le fil, mais ou est la matière ? même si tu gardes cette approche très théorique, il faut que tu décomposes beaucoup plus}
% \comment{(Florent) a revoir, trop de choses sont non définies}
% \comment{(Arnaud) morphogenetic}
% 
% Network Necessity : \comment{(Florent) je ne suis pas convaincu qu'il y a beaucoup de faits stylisés absolument impossibles à reproduire sans réseaux. mais les inclure (les réseaux) a des avantages que tu vas défendre}
% 
% \comment{(Florent) parfait je pense que c'est une bonne idée. peux tu néanmoins le justifier dans le contexte actuel d'incertitude et de développement durable cela me semble pertinent.}
%
% Stationarité :  \comment{(Florent) peux tu expliquer pourquoi tu souhaites cette propriété ?}
% \comment{(Florent) je ne suis pas convaincu par cette distinction sur les échelles : c'est sans doute différent mais pas qualitativement}
%
% sur le réseau de neurones : \comment{(Florent) tu es dans la méthode c'est un autre point de discussion}



\noun{Raffestin} souligne dans sa préface de~\cite{offner1996reseaux} qu'une théorie géographique articulant espaces, réseaux et territoires n'a jamais été formulée de manière cohérente, chaque approche ayant une vision réduite à certaines composantes seulement et ne visant pas à construire une théorie globalement cohérente. Une piste que nous proposons d'introduire ici est la conjonction des approches de la Théorie Evolutive et de la Morphogenèse, pour à la fois produire une théorie multi-scalaire et intégrant pleinement réseaux et territoires.





%%%%%%%%%%%
\subsection{Foundations}{Fondations}



%%%%%%%%%%%%%%
\subsubsection{Networked Human Territories}{Territoires Humains en Réseau}


\bpar{
Our first pillar has already been constructed before in the thematic exploration of the research subject. We rely on the notion of \emph{Human Territory} elaborated by \noun{Raffestin} as the basis for a definition of territorial systems. It permits to capture complex human geographical systems in their concrete and abstract characteristics and representation. For example, a metropolitan territorial system can be apprehended simply by the functional extent of daily commuting, or by the perceived or lived space of different populations, the choice depending on the precise question asked. Note that this approach to territory is a position and that other (possibly compatible) entries could be taken~\cite{murphy2012entente}. The concrete of this pillar in reinforced by the territorial theory of networks of \noun{Dupuy}, yielding the notion of networked human territory, as a human territory in which a set of potential transactional networks have been realized, which is in accordance with vision of the territory as networked places~\cite{champollion:halshs-00999026}. We make therein the assumption that real networks are necessary elements of territorial systems.
}{
Notre premier pilier a déjà été construit précédemment lors de l'exploration thématique en Chapitre~\ref{ch:thematic}. Nous nous basons sur la notion de \emph{Territoire Humain} élaborée par \noun{Raffestin} comme la base de la définition d'un système territorial. Elle permet de capturer les systèmes complexes géographiques humains dans l'ensemble de leur caractéristiques concrètes et abstraites, ainsi que dans leur représentations. Par exemple, un territoire métropolitain peut être appréhendé simplement par l'étendue fonctionnelle des flux pendulaires journaliers, ou par l'espace perçu ou vécu des différentes populations, le choix dépendant de la question précise à laquelle on cherche à répondre. Le territoire de \noun{Raffestin} devrait correspondre à un système cohérent de \emph{synergetic inter-representation networks}, qui est à la fois une théorie et un modèle pour la cognition spatiale des individus et des sociétés, construite par \emph{Portugali} et \emph{Haken} (voir~\cite{portugali2011sirn} pour une présentation synthétique). Elle postule que les représentations sont le produit du couplage fort entre les individus des cognitions et de leurs comportements individuels et collectifs. Cette approche au territoire est bien sûr un choix délibéré et que d'autres entrées, possiblement compatible, peuvent bien sûr être prises~\cite{murphy2012entente}. Le ciment de ce pilier est renforcé par la théorie territoriale des réseaux de \noun{Dupuy}, fournissant la notion de territoire humain en réseau, comme un territoire humain dans lequel un ensemble de réseaux transactionnels potentiels ont été réalisés, ce qui s'accorde par ailleurs avec les visions du territoire comme un lieu des réseaux~\cite{champollion:halshs-00999026}. Nous n'utiliserons pas les implications du développement de la notion de \emph{lieu}, celles-ci étant trop éparses (voir définition de \cite{hypergeo}), et à cause de la redondance avec le territoire dans la vision de lien complexe entre représentation et réalité physique. Nous ferons pour ce premier pilier l'hypothèse fondamentale, déjà introduite en chapitre~\ref{ch:thematic}, que les réseaux réels sont des éléments nécessaires des systèmes territoriaux.
}

% beware confusion between place and space : lieu en réseau ≠ lieu des réseaux ≠ espace en réseau ≠ espace des réseaux ???


%%%%%%%%%%%%%%%
\subsubsection{Evolutive Urban Theory}{Théorie Evolutive des Villes}



\bpar{
The second pillar of our theoretical construction is the Evolutive Urban Theory of \noun{Pumain}, closely linked to the complexity approach we take. This theory was first introduced in~\cite{pumain1997pour} which argues for a dynamical vision of city systems, in which self-organization is key. Cities are interdependent evolutive spatial entities whose interrelations produces the macroscopic behavior at the scale of city system. The city system is also designed as a network of city what emphasizes its view as a complex system. Each city is itself a complex system in the spirit of~\cite{berry1964cities}, the multi-scale aspect being essential in this theory, since microscopic agents convey system evolution processus through complex feedbacks between scales. The positioning within Complex System Sciences was later confirmed~\cite{pumain2003approche}. It was shown that this theory provide an interpretation for the origin of pervasive scaling laws, resulting from the diffusion of innovation cycles between cities~\cite{pumain2006evolutionary}. The aspect of resilience of system of cities, induced by the adaptive character of these complex systems, implies that cities are drivers and adapters of social change~\cite{pumain2010theorie}. Finally, path dependance yield non-ergodicity within these systems, making ``universal'' interpretations of scaling laws developed by physicists incompatible with evolutive urban theory~\cite{pumain2012urban}. The Evolutive Urban Theory was elaborated conjointly with models of urban systems: for example the Simpop2 model is an agent-based model taking into account economic processes, that simulates growth patterns on long time scales for Europe and the United States~\cite{doi:10.1177/0042098010377366}.
 The latest accomplishment of the evolutive theory relies in the output of the ERC project GeoDivercity, presented in~\cite{pumain2017urban}, that include both advanced technical (software OpenMole), thematic (knowledge from SimpopLocal and Marius models) and methodological (incremental modeling) progresses. We will interpret territorial systems following that idea of complex adaptive systems.
}{
Le second pilier de notre construction théorique est la théorie évolutive des villes de \noun{Pumain}, en relation étroite avec l'approche complexe que nous prenons de manière générale. Celle-ci a déjà été présenté en détails ainsi que ses implications explorées en Chapitre~\ref{ch:evolutiveurban}. Ici, cette théorie nous permet d'interpréter les systèmes territoriaux comme systèmes complexes adaptatifs avec les implications listées ci-dessus.
}





%%%%%%%%%%%
\subsubsection{Urban Morphogenesis}{Morphogenèse Urbaine}

% -> make a link between city systems and urban form/cityscape / territorial configurations

% Why morphogenesis is important : linked with modularity and scale -> if a submodule can be explained independantly (ie morphigeneis process is isolated), then we have the characteristic scale. then when size grows and interaction within city system -> can not explain alone (or with externalities ?) -> need a change in scale. ex. influecne of city system for size, activities; posiiton of an airport in a metropoltian region ; emergence of MCR.  ==> Assimptions to be tested with models ?.

% Alexander and Salingaros

% include transportation network, hierarchy and congestion in transport : Remy vs Benjamin (paper ? -> see with René)



\bpar{
The idea of morphogenesis was particularly underlined by \noun{Turing} in~\cite{turing1952chemical} % not exactly, already introduced before
 when trying to isolate simple chemical rules that could lead to the emergence of the embryo and its form. The morphogenesis of a system consists in self-consistent evolution rules that produce the emergence of its successives states, i.e. the precise definition of self-organization, with the additional property that an emergent architecture exists, in the sense of relations between form and function. Progresses towards the understanding of embryo morphogenesis (in particular the isolation of processes producing the differentiation of cells from an unique cell) has been made only recently with the use of Complexity Approaches in integrative biology~\cite{delile2016chapitre}. In the case of urban systems, the idea of urban morphogenesis, i.e. of self-consistent mechanisms that would produce the urban form, is more used in the field of architecture and urban design~\cite{hachi2013master} (as \noun{Alexander} generative grammar ``Pattern Language'' e.g.), in relation with theories of Urban Form~\cite{moudon1997urban}. This idea can be pushed into very small scales such as the building~\cite{whitehand1999urban} but we will use it more at a mesoscopic scale, in terms of land-use changes within an intermediate scale territorial system, in the same ontologies as Urban morphogenesis modeling literature (for example \cite{bonin2012modele} describes a model of urban morphogenesis with qualitative differentiation, whereas \cite{makse1998modeling} give a model of urban growth based on a mono-centric population distribution perturbed with correlated noises). The notion of morphogenesis will be important in our theory in link with modularity and scale. Modularity of a complex system consists in its decomposition into relatively independent sub-modules, and modular decomposition of a system can be seen as a way to disentangle non-intrinsic correlations~\cite{2015arXiv150904386K} (think of a block diagonalisation of a first order dynamical system). In the context of large-scale cyber-physical systems design and control, similar issues naturally raise and specific techniques are needed to scale up simple system control methods~\cite{2017arXiv170105880W}. The isolation of a subsystem yields a corresponding characteristic scale. Isolating possible morphogenesis processes imply a controlled isolation (controlled boundary conditions e.g.) of the considered system, corresponding to a modularity level and thus a scale. When self-consistent processes are not enough to explain the evolution of the system (with reasonable action on boundary conditions), a change of scale is necessary, caused by an underlying phase transition in modularity. The example of metropolitan growth is a good example: complexity of interactions within the metropolitan region will grow with size and diversity of functions leading to a change in scale necessary to understand processes. The emergence of an international airport will strongly influence local development, what corresponds to the significant integration within a larger system. The characteristic scales and processes for which these change occur will be precise questions to be investigated through modeling. It is interesting to remark that a territorial subsystem in which morphogenesis has a sense can be seen as an \emph{autopoietic system} in the extended sense of \noun{Bourgine} in~\cite{bourgine2004autopoiesis}, as a network of auto-reproducing processes\footnote{which are however not cognitive, making this auto-organized systems fortunately not alive in the sense of autopoietic and cognitive systems} regulating their boundary conditions, what emphasizes boundaries on which we will last insist.
}{
La notion de morphogenèse a été déjà explorée en profondeur et selon un point de vue interdisciplinaire en~\ref{sec:interdiscmorphogenesis}. Nous rappelons ici certains grands axes et dans quelles mesure ceux-ci contribuent à la construction de notre théorie. La morphogenèse a été particulièrement soulignée par \noun{Turing} dans~\cite{turing1952chemical} lorsqu'il proposait d'isoler des règles chimiques élémentaires qui pourraient mener à l'émergence de l'embryon et à sa forme. La morphogenèse d'un système consiste en des règles d'évolution auto-cohérentes qui produisent l'émergence de ses états successifs, i.e. la définition précise de l'auto-organisation, avec la propriété supplémentaire qu'une architecture émergente existe, au sens de relations causales circulaires entre la forme et la fonction. Les progrès vers la compréhension de la morphogenèse de l'embryon (en particulier l'isolation de processus particuliers induisant la différentiation de cellules à partir d'une unique) sont relativement récents grâce à l'application des approches complexes en biologie intégrative~\cite{delile2016chapitre}. Dans le cas des systèmes urbains, l'idée de morphogenèse urbaine, i.e. de mécanismes auto-cohérents qui produiraient la forme urbaine, est plutôt utilisé dans les champs de l'architecture et de l'urbanisme~\cite{hachi2013master} (comme e.g. la grammaire générative du ``Pattern Language'' d'\noun{Alexander}), en relation avec des théories de la forme urbaine~\cite{moudon1997urban}. Cette idée peut être poussée jusqu'à de très petites échelles comme celle du bâtiment~\cite{whitehand1999urban} mais nous l'utiliserons plus à une échelle mesoscopique, en termes de changements d'usage du sol à une échelle intermédiaire des systèmes territoriaux, avec des ontologies similaires à la littérature de modélisation de la morphogenèse urbaine (par exemple \cite{bonin2012modele} décrit un modèle de morphogenèse urbaine avec différentiation qualitative, tandis que \cite{makse1998modeling} donne un modèle de croissance urbaine basé sur une distribution monocentrique de la population perturbée par des bruits corrélés). La notion de morphogenèse sera importante dans notre théorie en lien avec la modularité et l'échelle. La modularité d'un système complexe consiste en sa décomposition en sous-modules relativement indépendants, et la décomposition modulaire d'un système peut être vue comme un moyen de supprimer les correlations non intrinsèques~\cite{2015arXiv150904386K} (pour donner une image, penser à une diagonalisation par blocs d'un système dynamique du premier ordre). Dans le cadre de la conception et du contrôle de systèmes cyber-sociaux à grande échelle, des problèmes similaires surgissent naturellement et des techniques spécifiques sont nécessaires pour le passage à l'échelle des techniques simple de contrôle~\cite{2017arXiv170105880W}. L'isolation d'un sous-système fournit une échelle caractéristique correspondante. Isoler des processus de morphogenèse possibles implique une extraction contrôlée (conditions au bord contrôlées par exemple) du système considéré, ce qui correspond à un niveau de modularité et donc à une échelle. Quand des processus auto-cohérents ne sont pas suffisants pour expliquer l'évolution d'un système (dans des variations raisonnables des conditions initiales), un changement d'échelle est nécessaire, causé par une transition de phase implicite dans la modularité. L'exemple de la croissance métropolitaine en est une très bonne illustration : la complexité des interactions au sein de la région métropolitaine sera croissante avec sa taille et la diversité des fonctions urbaines, ce qui conduit à un changement de l'échelle nécessaire pour comprendre les processus. L'émergence d'un aéroport international pourra dans certains cas influencer fortement le développement local, ce qui correspondra à une intégration significative dans un système plus vaste. Les échelles caractéristiques et la nature des processus pour lesquels ces changements ont lieu peuvent être des questions précisément approchées par l'angle de la modélisation. Il est important de noter qu'un sous-système territorial pour lequel la morphogenèse prend sens et dont les frontières sont bien définies peut être vu comme un \emph{système auto-poiétique} au sens étendu de \noun{Bourgine} dans~\cite{bourgine2004autopoiesis}, i.e. comme un réseau de processus qui s'auto-reproduisent\footnote{qui ne sont toutefois pas cognitifs, ne rendant pas ces systèmes morphogénétiques vivants au sens de auto-poiétique et cognitif} en régulant leur conditions aux bords, ce qui souligne la notion de frontière sur laquelle nous allons finalement nous attarder.
}




% transition : Bourgine autopoiesis -> importance of boundaries -> link to Holland.

%%%%%%%%%%%
\subsubsection{Co-evolution}{Co-évolution}

% other insight : Holland Signal and Boundaries, ecological niche etc. : contextualize within this framework, clarify definition of co-evolution


\bpar{
Our last pillar is a clarification of the notion of \emph{co-evolution}, on which \noun{Holland} shed light through an approach of complex adaptive systems by a theory of CAS as signal processing agents operating thanks to their boundaries~\cite{holland2012signals}. In this theory, complex adaptive systems form aggregates at diverse hierarchical levels, that correspond to different level of self-organization, and boundaries are vertically and horizontally intricate in a complex way. That approach introduces the notion of \emph{niche} as a relatively independent subsystem in which ressources circulate (the same way as network communities): numerous illustrations are given such as economical niches or ecological niches. Agents within a niche are said to be \emph{co-evolving}. Co-evolution thus means strong interdependences (implying circular causal processes) and a certain independence regarding the exterior of the niche. The notion is naturally flexible as it will depend on ontologies, resolution, thresholds etc. considered to define the system. This concept is easily transmissible to the evolutive urban theory and converges with the notion of co-evolution described by \noun{Pumain}: co-evolving agents in a system of cities consist in a niche with its flows, signals and boundaries and thus co-evolving entities in the sense of \noun{Holland}. This notion will be important for us in the definition of territorial subsystems and their coupling.
}{
Notre dernier pilier consiste en une clarification de la notion de \emph{co-evolution}, sur laquelle \noun{Holland} apporte un éclairage pertinent à travers son approche des systèmes complexes adaptatifs (CAS) par une théorie des CAS comme agents dont la propriété fondamentale est de traiter des signaux grâce à leur frontières~\cite{holland2012signals}. Dans cette théorie, les systèmes complexes adaptatifs forment des agrégats à différents niveaux hiérarchiques, qui correspondent à différents niveaux d'auto-organisation, et les frontières sont intriquées horizontalement et verticalement de manière complexe. Cette approche introduit la notion de \emph{niche} comme un sous-système relativement indépendant au sein duquel les ressources circulent (de la même façon que des communautés dans un réseau) : de nombreuses illustrations telles les niches écologiques ou économiques peuvent être données. Les agents au sein d'une niche sont dits en \emph{co-évolution}. Empiriquement, les résultats obtenus témoignant d'une co-évolution à l'échelle mesoscopique comme en~\ref{sec:causalityregimes}, confirment l'existence de niches pour certains aspects des systèmes territoriaux. La co-évolution implique ainsi de fortes interdépendances (impliquant des processus causaux circulaires) et une certaine indépendance au regard de l'extérieur de la niche. La notion est naturellement flexible puisqu'elle dépendra des ontologies, de la résolution, des seuils, etc. que l'on considère pour définir le système. Nous postulons vu les indices d'existence obtenus dans les résultats empiriques, mais aussi les modèles reproduisant les processus de manière crédible sous une hypothèse d'isolation raisonnable, que ce concept peut se transmettre à la théorie évolutive urbaine et correspond à la notion de co-évolution décrite par \noun{Pumain} : des agents co-évolutifs dans un système de villes consistent en une niche et ses flots, signaux et limites et sont donc des entités co-évolutives au sens de \noun{Holland}. Cette notion sera importante pour nous dans la définition des sous-systèmes territoriaux et de leur couplage. Nous gardons à l'esprit les potentialités et limitation du parallèle entre systèmes biologiques et systèmes sociaux décrits en~\ref{sec:epistemology}.
}







%%%%%%%%%%%
%\subsection[A theory of co-evolutive networked territorial systems][Une théorie des systèmes territoriaux co-évolutifs en réseau]{Synthesis: a theory of co-evolutive networked territorial systems}{Synthèse : une théorie des systèmes territoriaux co-évolutifs en réseau}
\subsection{A theory of co-evolutive networked territorial systems}{Une théorie des systèmes territoriaux co-évolutifs en réseau}



\bpar{
We synthesize our pillars as a short self-consistent geographical theory of territorial systems in which networks play a central role in the co-evolution of components of the system. See the foundation subsection for definitions and references. The formulation is intended to be minimalistic.
}{
Nous synthétisons les différents piliers en une théorie géographique autonome des systèmes territoriaux pour lesquels les réseaux jouent un rôle central pour la co-évolution des composantes du système. Pour les définitions des termes et les références, se référer à la section précédente. La formulation ici est voulue minimaliste.
}


\medskip


\bpar{
\begin{definition}
\textbf{ - Territorial System.} A territorial system is a set of networked human territories, i.e. human territories in and between which real networks exist.
\end{definition}
}{
\begin{definition}
\textbf{ - Système Territorial.} Un système territorial est un ensemble de territoires humains en réseau, c'est à dire des territoires humains au sein desquels et entre lesquels des réseaux réels existent.
\end{definition}
}

\comment[FL]{a mettre bien plus haut: c'est assez fondamental}

\medskip


\bpar{
At this step complexity and dynamical evolutive characters of territorial systems are implied but not an explicit part of the theory. We will assume to simplify a discrete definition of temporal, spatial and ontological dimensions under modularity and local stationarity assumptions.
}{
Le territoire est bien un élément du système territorial, qui de manière plus générale connecte différents territoires par les réseaux. A cette étape la complexité et le caractère évolutif et dynamique des systèmes territoriaux sont impliqués par les partis pris mais pas une partie explicite de la théorie. We supposerons pour simplifier une définition discrète des dimensions temporelles, spatiales et ontologiques, sous des hypothèses de modularité et de stationnarité locale. Cet aspect, à la fois pour le discret et la stationnarité, correspond à une simplification ontologique de la supposition d'une ``échelle minimale'' à laquelle les sous-systèmes fournissent une décomposition modulaire simple du système global. Elle reflète nos conclusions empiriques obtenues en Chapitre~\ref{ch:micro} et les modèles développés par la suite. On suppose également ergodicité locale, pour obtenir grâce à la démonstration proposée en~\ref{sec:staticcorrelations} la propriétés de non-ergodicité globale typique des systèmes urbains.
}
    

\medskip

% assumption : existence of scales

\bpar{
\begin{proposition}
\textbf{ - Discrete scales.} Assuming a discrete modular decomposition of a territorial system, the existence of a discrete set $(\tau_i,x_i)$ 
of temporal and functional scales for the territorial system is equivalent to the local temporal stationarity of a random dynamical system specification of the system.
\end{proposition}
}{
\begin{proposition}
\textbf{ - Echelle discrètes.} Supposant une décomposition modulaire discrète d'un système territorial, l'existence d'un ensemble discret $(\tau_i,x_i)$ d'échelles temporelles et fonctionnelles pour le système territorial est équivalent à la stationnarité temporelle locale d'une spécification par système dynamique stochastique du système. % Q here : does the master eq needs to be stochastic ?
\end{proposition}
}


\bpar{
\begin{proof}
\textbf{(Sketch of).} We underlie that any territorial system can be represented by random variables, what is equivalent to have well defined objects and states and use the Transfer Theorem on events of successive states. If $X=(X_j)$ is the modular decomposition, we have necessarily quasi-independence of components in the sense that $\Covb{dX_j}{dX_{j'}}\simeq 0$ at any time. General stationarity transitions induce modular transitions that are kept or not depending if they correspond to an effective transition within the subsystem, what provide temporal scales as characteristic times of sub-dynamics. Functional scales are the corresponding extent in the state space.\qed
\end{proof}
}{
\textbf{Preuve (Tentative).} Nous partons de l'hypothèse que tout système territorial peut être représenté par un ensemble de variables aléatoires, ce qui revient à avoir des objets et états bien définis et utiliser le Théorème de Transfert sur les évènements des états successifs. Si $X=(X_j)$ est la décomposition modulaire, on a nécessairement quasi-indépendance des composantes au sens que $\Covb{dX_j}{dX_{j'}}\simeq 0$ à tout moment.\comment[FL]{a discuter} Les transitions de stationnarité globales induise des transitions dans chaque module, qui sont conservées si elles correspondent effectivement à un transition dans le sous-système. On obtient ainsi les échelles temporelles comme temps caractéristiques des sous-dynamiques. Les échelles fonctionnelles sont les étendues correspondantes dans l'espace d'état.\qed
}




\medskip


\bpar{
This proposition induce a discrete representation of system dynamics in time. Note that even in the case of no modular representation, the system as a whole will verify the property. This definition of scales allows to explicitly introduce feedback loops and thus emergence and complexity, making our theory compatible with the evolutive urban theory.
}{
Cette proposition postule une représentation des dynamiques du système dans le temps. On peut noter que même en l'absence de représentation modulaire, le système dans son ensemble vérifiera la propriété. Cette définition des échelles permet d'introduire explicitement des boucles de rétroaction, puisqu'on peut par exemple conditionner l'évolution d'une échelle à celle d'une autre qui la contient, et ainsi l'émergence et la complexité, rendant la théorie compatible avec la théorie évolutive urbaine.
}



\bpar{
\begin{assumption}
\textbf{ - Scales and Subsystems intrication. } Complex networks of feedbacks exist both between and inside scales, what impose the existence of weak emergence~\cite{bedau2002downward}. Furthermore a horizontal and vertical hierarchical imbrication of boundaries is not the rule.
\end{assumption}
}{
\begin{assumption}
\textbf{ - Imbrication des échelles et des sous-systèmes. } Des réseaux complexes de retroaction existent à la fois entre et à l'intérieur des échelles~\cite{bedau2002downward}. De plus, un emboîtement horizontal et vertical des limites ne sera généralement pas hiérarchique.
\end{assumption}
}

% co-evolution

\bpar{
Within these complex subsystems intrications we can isolate co-evolving components using morphogenesis. The following proposition is a consequence of the equivalence between the independence of a niche and its morphogenesis. Morphogenesis provides the modular decomposition (local stationarity assumed) needed for the existence of scale, giving minimal vertically (scale) and horizontally (space) independent subsystems.
}{
Au sein de ces imbrications de sous-systèmes nous pouvons isoler des composantes en co-évolution en utilisant la morphogenèse. La proposition suivante est une conséquence de l'équivalence entre l'indépendance d'une niche et sa morphogenèse. La morphogenèse fournit la décomposition modulaire (sous hypothèse de stationnarité locale) nécessaire pour l'existence de l'échelle, donnant des sous-systèmes minimaux indépendants de manière verticale (échelle) et horizontale (espace).
}


\bpar{
\begin{proposition}
\textbf{ - Co-evolution of components. } Morphogenesis processes of a territorial system are an equivalent formulation of the existence of co-evolutive subsystems.
\end{proposition}
}{
\begin{proposition}
\textbf{ - Co-évolution des composantes. } Les processus morphogénétiques d'un système territorial sont une formulation équivalente de l'existence de sous-systèmes co-évolutifs.
\end{proposition}
}



% importance of nws as necessary subcomponents
%  maybe where we diverge from Denise theory ?


\bpar{
Finally we make a key assumption putting real networks at the center of co-evolutive dynamics, introducing their necessity to explain dynamical processes of territorial systems.
}{
Nous formulons finalement la dernière hypothèse clé qui met les réseaux réels au centre des dynamiques co-évolutives, introduisant leur nécessité pour expliquer les processus dynamiques des systèmes territoriaux.
}


\bpar{
\begin{assumption}
\textbf{ - Necessity of Networks. } Network evolution can not be explained only by the dynamics of other territorial components and reciprocally, i.e. co-evolving territorial subsystems include real networks. They can thus be at the origin of regime changes (transition between stationarity regimes) or more dramatic bifurcations in dynamics of the whole territorial system.
\end{assumption}
}{
\begin{assumption}
\textbf{ - Nécessité des réseaux. } L'évolution des réseaux ne peut pas être expliquée simplement par la dynamique des autres composantes territoriales et réciproquement, i.e. les sous-systèmes territoriaux co-évolutifs contiennent les réseaux réels. Ceux-ci peuvent ainsi être à l'origine de changements de régime (transitions entre régimes stationnaires) ou de bifurcations plus conséquentes dans les dynamiques de l'ensemble du système territorial.
\end{assumption}
}




\subsection{Contextualization}{Contextualisation}


\bpar{
On long time scale, an overall co-evolution has been shown for the French railway network by~\cite{bretagnolle:tel-00459720}. At smaller scales it is less evident (debate on structural effects) but we postulate that co-evolution effects are present at any scale. Regional examples may illustrate that : Lyon has not the same dynamical relations with Clermont than with Saint-Etienne and network connectivity has necessarily a role in that (among intrinsic interaction dynamics and distance). At a smaller scale, we think that effects are even less observable, but precisely because of the fact that co-evolution is stronger and local bifurcations will occur with stronger amplitude and greater frequency than in macroscopic systems where attractors are more stable and stationarity scales greater. We will try to identify bifurcation or phase transitions in toy models, hybrid models and empirical analysis, at different scales, on different case studies and with different ontologies.
}{
Sur de longues échelles temporelles, une co-évolution globale a été montrée pour le systèmes ferroviaire français par~\cite{bretagnolle:tel-00459720}. A de plus petites échelles celle-ci est moins évidente (débat sur les effets structurants) mais nous supposons la présence d'effets co-évolutifs à toutes les échelles. Des exemples régionaux peuvent illustrer ce fait : Lyon n'a pas les mêmes relations dynamiques avec Clermont qu'avec Saint-Etienne, et la connectivité de réseau a probablement un rôle à y jouer (parmi les effets des dynamiques intrinsèques des interactions, et de la distance par example). A une plus petite échelle encore, nous partons du principe que les effets sont encore moins observables, mais précisément à cause du fait que la co-évolution est plus forte et les bifurcations locales se produisent avec une plus grande amplitude et une plus grande fréquence que dans les systèmes macroscopiques où les attracteurs sont plus stables et les échelles de stationnarité plus grandes. Nous pour cela que nous avons tenté d'identifier des bifurcations ou des transitions de phase dans des modèles jouets, des modèles hybrides, et des analyses empiriques, à différentes échelles, sur différents cas d'études et avec différentes ontologies. 
}



\bpar{
One difficulty in our construction is the stationarity assumption. Even if it seems a reasonable assumption on large scales and has already been observed in empirical data~\cite{sanders1992systeme}, we shall verify it in our empirical studies. Indeed, this question is at the center of current research efforts to apply deep learning techniques to geographical systems: \noun{Bourgine} has recently developed a framework to extract patterns of Complex Adaptive Systems. Using a representation theorem~\cite{knight1975predictive}, any discrete stationary process is a \emph{Hidden Markov Model}. Given the definition of a causal state as $\Pb{future | A} = \Pb{future | B}$, the partition of system states induced by the corresponding equivalence relations allows to derive a \emph{Recurrent Network} that is sufficient to determine the next state of the system, as it is a \emph{deterministic} function of previous state and hidden states~\cite{shalizi2001computational}: $(x_{t+1},s_{t+1}) = F\left[(x_t,s_t)\right]$. The estimation of Hidden States and of the Recurrent Function thus captures through deep learning entirely dynamical patterns of the system, i.e. full information on its dynamics and internal processes. The issues are then if the stationarity assumption be tackled through augmentation of system states, and if heterogeneous and asynchronous data can be used to bootstrap long time-series necessary for a correct estimation of the neural network. These issue are related to the stationarity assumption for the first and to non-ergodicity for the second.
}{
Une difficulté dans notre construction est l'hypothèse de stationnarité locale, qui est essentielle pour formuler des modèles à l'échelle correspondante. Même si cela paraît une hypothèse raisonnable à plusieurs échelles et a déjà été observé des des données empiriques~\cite{sanders1992systeme}, nous devrons le vérifier dans nos études empiriques. En effet, cette question est au centre des efforts de recherche courants pour appliquer les techniques d'apprentissage profond aux systèmes géographiques : \noun{Bourgine} a récemment développé un cadre pour extraire des motifs des systèmes complexes adaptatifs. En utilisant un théorème de représentation~\cite{knight1975predictive}, tout processus stationnaire discret est un \emph{Modèle de Markov Caché}. Etant donné la définition d'un état causal comme $\Pb{future | A} = \Pb{future | B}$, la partition des états du système par la relation d'équivalence correspondantes permet de produire un \emph{Réseau Récurrent} qui est suffisant pour déterminer l'état suivant du système, puisqu'il s'agit d'une fonction \emph{déterministe} des états précédents et des états cachés~\cite{shalizi2001computational} : $(x_{t+1},s_{t+1}) = F\left[(x_t,s_t)\right]$. L'estimation des états cachés et de la fonction récurrente capture ainsi entièrement par apprentissage profond le comportement dynamique du système, i.e. l'information complète sur ses dynamiques et les processus internes. Les questions sont ensuite si les hypothèses de stationnarité peuvent être réglés par augmentation des états du système, et si des données hétérogènes et asynchrones peuvent être utilisées pour initialiser des séries temporelles assez longues pour une estimation correcte du réseau de neurones ou de tout autre type d'estimateur. Ces questions sont reliées à l'hypothèse de stationnarité pour la première et à la non-ergodicité pour la seconde.
}






\stars










