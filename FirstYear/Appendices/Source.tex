

% Chapter 

\chapter{Architecture and Sources for Algorithms and Models of Simulation} % Chapter title

\label{app:code} % For referencing the chapter elsewhere, use \autoref{ch:name} 

%----------------------------------------------------------------------------------------


% do not list all codes, but roughly gives architectures overview
%   and links to git repo

% : script that generates this directly from metadata files ? INCLUDING temporal statistics from git

% idem for work stats ! from git history

% Q : current state of programs ? -> frozen state on specific branch for each model -> could use metafig that way also ?

\headercit{You must not be afraid of putting code in your thesis, code is not dirty}{Alexis Drogoul}{}



And yet it is. It makes no sense to put code listings in the core of the text if there is no particular algorithmic detail that requires attention. As soon as implementation biases are avoided, architecture and source for a computational model should be independent from its formal description (but provided along model description with source code as already mentioned before). We give in this appendix architectural details on main models of simulation or algorithms we used. Langage and size (in code lines) are provided, along with architectural remarkable features. See \url{https://github.com/JusteRaimbault/CityNetwork/tree/master/Models} for all models, empirical analysis and small experiments. The following reports are partially generated automatically using experimental tools aimed at workflow improvement.


%----------------------------------------------------------------------------------------

%\newpage

\section{Algorithmic Systematic Review}

\paragraph{Objective}

Implement systematic literature review algorithm.

\paragraph{Location}

\url{https://github.com/JusteRaimbault/CityNetwork/tree/master/Models/Biblio/AlgoSR/AlgoSRJavaApp}

\paragraph{Characteristics}

\begin{itemize}
\item Language : \texttt{Java}
\item Size : 7116
\end{itemize}

\paragraph{Particularities}

\begin{itemize}
\item HashConsing used for unique bibliography object, specific hashcode switching if id available or only titles (proceed to lexical distance comparison in that latest case).
\item API to cortext currently being replaced by Python scripts.
\end{itemize}

\paragraph{Architecture}

Classical object oriented, see code.

\paragraph{Additional scripts}

\texttt{R} for result exploration and visualization.

%----------------------------------------------------------------------------------------


\section{Indirect Bibliometrics}

\paragraph{Objective}

Hypernetworks analysis of cybergeo journal.

\paragraph{Location}

CLOSED (shared repository).

\url{https://github.com/JusteRaimbault/CityNetwork/tree/master/Models/Biblio/AlgoSR/AlgoSRJavaApp} for common Java part.

\paragraph{Characteristics}

\begin{itemize}
\item Language : \texttt{Python}, \texttt{R} and \texttt{Java}.
\item Size : -
\end{itemize}


\paragraph{Particularities}

Polyglot 

\paragraph{Architecture}

See schema chapter 3.

\paragraph{Additional scripts}

-



%----------------------------------------------------------------------------------------


\section{Density Urban Growth}

\paragraph{Objective}

Simple density urban growth model.

\paragraph{Location}

\url{https://github.com/JusteRaimbault/CityNetwork/tree/master/Models/Synthetic/Density}

\paragraph{Characteristics}

\begin{itemize}
\item Language : \texttt{NetLogo} then \texttt{scala}.
\item Size : 4355
\end{itemize}


\paragraph{Particularities}

Morphological indicators in scala implemented with Fast Fourier transform ; with R communication in NetLogo.

\paragraph{Architecture}

Nothing particular.

\paragraph{Additional scripts}

\texttt{R} for result exploration and morphological analysis.

\texttt{oms} for model exploration.



%----------------------------------------------------------------------------------------


\section{Correlated data generation}

\paragraph{Objective}

Weak coupling of density generation and network generation.

\paragraph{Location}

\url{https://github.com/JusteRaimbault/CityNetwork/tree/master/Models/Synthetic/Network_20151229}

\paragraph{Characteristics}

\begin{itemize}
\item Language : \texttt{NetLogo} (network) and \texttt{scala}.
\item Size : 3188
\end{itemize}


\paragraph{Particularities}

Network heuristic easier to implement and explore in netlogo

\paragraph{Architecture}

OpenMole allows coupling between modules through exploration script.

\paragraph{Additional scripts}

\texttt{R} for result exploration.

\texttt{oms} for model exploration.





%----------------------------------------------------------------------------------------


\section{Lutecia Model}

\paragraph{Objective}

Implementation of Lutecia model, chapter 7.

\paragraph{Location}

\url{https://github.com/JusteRaimbault/CityNetwork/tree/master/Models/Governance/MetropolSim/Lutecia}

\paragraph{Characteristics}

\begin{itemize}
\item Language : \texttt{NetLogo}
\item Size : 4791
\end{itemize}


\paragraph{Particularities}

Shortest path dynamical programming using matrices.

\paragraph{Architecture}

Pseudo object architecture in agent environment.

\paragraph{Additional scripts}

\texttt{R} for result exploration.

\texttt{oms} for model exploration.




%----------------------------------------------------------------------------------------


\section{Network analysis}

\paragraph{Objective}

Simplification of european road network

\paragraph{Location}

\url{https://github.com/JusteRaimbault/CityNetwork/tree/master/Models/StaticCorrelations}

\paragraph{Characteristics}

\begin{itemize}
\item Language : \texttt{R}, \texttt{Shell}, \texttt{PostgreSQL}
\item Size : 505
\end{itemize}


\paragraph{Particularities}

Handling of large size databases imposes sequential processing ; use of external program \texttt{osmosis} for conversion from \texttt{osm} data to pgsql.

\paragraph{Architecture}

Shell script lead manoeuvers.

\paragraph{Additional scripts}

-



