


% Chapter 

\chapter{Tools and Workflow for an open Reproducible Research} % Chapter title

\label{app:workflow} % For referencing the chapter elsewhere, use \autoref{ch:name} 

%----------------------------------------------------------------------------------------


\headercit{Open for Discovery}{PLoS}{}


%% 
%   Technical elements / workflow to be included :
%
%   - NLaggregator : useful ?
%   - NLDocumentation : yes
%   - git usage
%   - towards a git-compatible metafig ? / metadata-handler 
%   !! importance of metadata for transparency/reproducibility
%
%



%\section{Context}


\bigskip

We briefly evoke here tools or workflows currently under development or testing, aimed at easing an open reproducible research and making it more transparent.


\section{NetLogo documentation generator}

Documentation generation is central for reproducibility as it can automatize implementation description. NetLogo does not provide a documentation generator and we are thus currently writing a Doxygen wrapper for NetLogo code, that basically consists in transforming NetLogo code into Java code and parsing documentation comment blocks. An experimental version is available at \url{https://github.com/JusteRaimbault/CityNetwork/tree/master/Models/Doc}.



\section{git as a reproducibility tool}

The use if \texttt{git} as a reproducibility and transparency tool was emphasized in~\cite{ram2013git} (for various reasons such as exact history tracing, easy cloning, past commit branching). It furthermore can help individual workflow for advantages such as automatic backup, organisation, experiments tracking. We use it actively and develop extensions for it.


\section{git-data}

\texttt{git-data} is a shell based (experimental) git extension, available at \url{https://github.com/JusteRaimbault/gitdata}, that allows automatized backup of large file within a git repository, their transparent integration in ignored files and the creation of symbolic links for a transparent local use.


\section{Towards a git-compatible figures metadata handler}

The issue of meta-data for figures is a crucial issue, as it is often difficult to keep a trace of all parameter values that have generated it, along with the corresponding code. Tricks may furthermore happen in script environments such as R or python when variables are accidentally modified without code modification. Keeping an exhaustive trace of the exact dataset, code and history that has generated a precise figure is a necessary condition for exact reproducibility. We are elaborating a git-compatible tool that would automatically handle these metadata, for example by branching and associating the unique commit hash to the figure. To become not an organizational burden nor a repository perturbation, we must still make some experiments. The final idea would be to have under each figure a unique identifier linking to the associated reproducting environment.


\section{TorPool}

TorPool is a java based Tor wrapper available with an api (currently only java, R version projected) at \url{https://github.com/JusteRaimbault/TorPool}. It allows among other purposes tricky data retrieval.

