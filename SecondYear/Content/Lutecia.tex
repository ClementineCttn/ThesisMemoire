



% Chapter 

\chapter{Transportation Governance Modeling}{Modélisation de la Gouvernance du Système de Transport} % Chapter title

\label{ch:lutetia} % For referencing the chapter elsewhere, use \autoref{ch:name} 


%----------------------------------------------------------------------------------------

This single section chapter is differentiated from the previous one as it makes a step further towards more complex models. A toy-model introducing governance processes is described. Such exploration logically enters our theoretical framework to try to validate or invalidate the network necessity assumption : if non-linear necessary processes are highlighted and validated against stylized facts, it argues towards the validation of this assumption. 

Other targeted projects such as the exploration of an hybrid macro-economic/accessibility-based model to explore transportation companies line implementation strategies are still at the state of ideas and are not described here.



%----------------------------------------------------------------------------------------


\section[The Lutecia Model]{Taking Governance into account in Network Production Processes : The Lutecia Model}{Le Modèle Lutecia}



\todo{insert and translate Lutecia paper}






