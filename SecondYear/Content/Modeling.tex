

% Chapter 

\section{Modeling}{Modélisation} % Chapter title

\label{ch:modeling} % For referencing the chapter elsewhere, use \autoref{ch:name} 





%----------------------------------------------------------------------------------------

\headercit{Do or do not. There is no try.}{Yoda}{}


\bigskip


One does not simply \emph{try} to model something. On that point personal experience confirms indeed that point, as I remember as an early Master student giving in to the call of incautious agent-based modeling, \comment{(Florent) pas forcément approprié dans une thèse}
 naively thinking that integrated models of any aspect of an urban system could be constructed, producing numerous NetLogo code lines to build a gaz factory with unfounded internal processes, an extremely poor external validation and no internal validation. This was a try and therefore a step towards the dark side of models bricolage. The construction of a computational model of simulation is a rigorous exercise that one can not improvise, as much as statistical modeling. Recent progresses in the field~\cite{banos2013pour} help to that purpose, and modular model construction and validation is one tool useful to avoid becoming lost in shady places.

We propose in this chapter simple modeling experiments, conceived to be preliminaries for more elaborated tests of our theory. We begin with a simple diffusion-aggregation model of urban growth as a relatively small scale. Beginning with simple assumptions does not mean a non-rigorous exploration of the model, that is therefore explored and calibrated on real data. The fact that we reproduce existing urban forms \comment{(Florent) oui mais ce n'est pas du tout un problème pour ta thèse. Ta thèse porte sur FU $\leftrightarrow$ Réseaux ; le fait qu'on puisse faire des modèles forme urbaine purs (avec mécanismes abstraits) et idem pour les réseaux d'ailleurs ne change rien à l'affaire. tout tes modèles ne doivent pas viser à reproduire FU \& réseau mais à reproduire FU $\leftrightarrow$ Réseaux}
 without the use of networks suggest either the total absence of network influence at this scale, or its very strong influence yielding apparent random effects that disappear in average calibration. We propose then to simply couple this model with a network generation heuristic in order to study feasible correlations between morphology and network. The absence of coupled calibration \comment{(Florent) pourquoi ? remarque tu dis le contraire là ?}
  avoids to draw empirical conclusion but the method is satisfying in itself as it permits the generation of synthetic territorial configurations where correlation structure is controlled. We finally describe a project of benchmark of diverse heuristic models for network generation.
















