

% Chapter 

\section{Empirical Analysis : Insights from Stylized Facts}{Analyse Empirique} % Chapter title

\label{ch:empirical} % For referencing the chapter elsewhere, use \autoref{ch:name} 

%----------------------------------------------------------------------------------------


\headercit{Mais ce n'est pas une question d'{\^a}ge, de chiffres et de stats\\ Moi je te parle surtout de rage, de kif et d'espoir}{Youssoupha}{\textit{, Esperance de Vie}}


\bigskip


%  plan : 

%  1) static morphological analysis : requires a formal link between temporal and spatial correlations ?  -- typology etc can already be interesting --


%  2) presentation of BP case study

%  3) base Bien

%  4) Work with Solène



%----------------------------------------------------------------------------------------


As this quote suggests, a purely quantitative view of the world makes no sense without qualitative counterbalancing. More precisely, we argue that the \textit{clich{\'e}} of an opposition between quantitative and qualitative analysis is an illusion. No distinct boundary exists between both. We propose to call quantitative any process involving computation by a Turing machine, whereas the qualitative will be for us the modeling design process and its interpretations. \comment{(Florent) je ne sais pas si je rangerais l'interprétation dans le qualitatif ; ok pour dire (même si connait rien en machine de Turing) que certaines observations via ``Turing'' peuvent s'appeler quantitatives. mais dans un cas comme dans l'autre, ensuite, il faut interpreter}
 Therefore both are necessarily closely interlaced in any of our approaches. In particular concerning the construction and the validation or refutation of our theory, empirical analysis on real case studies, implying the extraction and qualification of stylized facts, follows that schema.

% articulation with theoretical questions
% articulation with modeling

\comment{(Florent) Entre cette phrase (next) et le détail projet par projet, il serait bien que tu te positionnes sur le cadre analytique général, justement comme tu l'introduis en parlant des objets et des échelles}

We propose in this chapter various empirical analysis on different objects at different scales. A first section begins the examination of static spatial correlations between morphological measures of population density and road network measures on Europe at a 500m resolution. Applying last section of the methodological chapter should provide information on typical spatial scales of interaction between these indicators \comment{(Florent) à quel niveau se situe l'indicateur ?}
 of territory and network and on dynamical correlations between these. These computation furthermore provide empirical measures on which one model will be calibrated. We then describe a roadmap for statistical analysis on dynamical data of interactions for Bassin Parisien in the last fifty years. An other project using Real Estate transaction data for Parisian Metropolitan Region aim at seeking early warning of network breakdowns. We finally describe potential analyses on South African historical data. \comment{(Florent) ah bon et maintenant la Chine ? :) }














