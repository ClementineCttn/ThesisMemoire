

% Chapter 




\chapter{Interactions between Networks and Territories} % Chapter title

\label{ch:thematic} % For referencing the chapter elsewhere, use \autoref{ch:name} 


%%  Thematic chapter framing geographically the subject.
%%   and reviewing state of the art
%%   and why modeling : evolutive theory of urban systems etc ; multimodeling simfamily etc
%%  
%%   Q  : example to introduce theory ?
%
%   Modelography.  (non-exhaustive) : classification according to purpose, theme, scale, etc.
%   Why dynamic models of ``co-evolution''  ?
%   definition of terms, contextualisation, etc.  (le what/where d'Arnaud ; ontology de Anne)



%----------------------------------------------------------------------------------------

%\headercit{If you are embarrassed by the precedence of the chicken by the egg or of the egg by the chicken, it is because you are assuming that animals have always be the way they are}{Denis Diderot}{\cite{diderot1965entretien}}

\headercit{Si la question de la priorit{\'e} de l'\oe{}uf sur la poule ou de la poule sur l'\oe{}uf vous embarrasse, c'est que vous supposez que les animaux ont {\'e}t{\'e} originairement ce qu'ils sont {\`a} pr{\'e}sent.}{Denis Diderot}{\cite{diderot1965entretien}}
%\headercit{Si la question de la priorit{\'e} de l'\oe{}uf sur la poule ou de la poule sur l'\oe{}uf vous embarrasse, c'est que vous supposez que les animaux ont {\'e}t{\'e} originairement ce qu'ils sont {\`a} pr{\'e}sent.}{Denis Diderot}{\cite{diderot1965entretien}}


\bigskip


\bpar{
This analogy is ideal to evoke the questions of causality and processes in territorial systems. When trying to tackle naively our preliminary question, some observers have qualified the identification of causalities in complex systems as ``chicken and egg'' problems : if one effect appears to cause another and reciprocally, how can one disentangle effective processes ? This vision is often present in reductionist approaches that do not postulate an intrinsic complexity in studied systems. The idea that Diderot suggests is the notion of \emph{co-evolution} that is a central phenomenon in evolutive dynamics of Complex Adaptive Systems as \noun{Holland} develops in~\cite{holland2012signals}. He links the notion of emergence (that is ignored in a reductionist vision), in particular the emergence of structures at an upper scales from the interactions between agents at a given scale, materialized generally by boundaries, that become crucial in the coevolution of agents at any scales : the emergence of one structure will be simultaneous with one other, each exploiting their interrelations and generated environments conditioned by their boundaries. We shall explore these ideas in the case of territorial systems in the following.
}{
Cette analogie est idéale pour introduire les notions de causalité et de processus dans les systèmes territoriaux. En voulant traiter naïvement des questions similaires à notre question de recherche préliminaire, certains on qualifiés les causalités au sein de systèmes complexes comme un problème ``de poule et {\oe}uf'' : si un effet semble causer l'autre et réciproquement, comment est-il possible d'isoler les processus correspondants ? Cette vision est souvent présente dans les approches réductionnistes qui ne postulent pas une complexité intrinsèque au sein des systèmes étudiés. L'idée suggérée par \noun{Diderot} est celle de \emph{co-evolution} qui est un phénomène central dans les dynamiques évolutionnaires des Systèmes Complexes Adaptatifs comme \noun{Holland} élabore dans~\cite{holland2012signals}. Il fait le lien entre la notion d'émergence (ignorée dans les approches réductionnistes), en particulier l'émergence de structures à une plus grand échelle par les interactions entre agents à une échelle donnée, en général concrétisée par un systèmes de limites, qui devient cruciale pour la co-évolution des agents à toutes les échelles : l'émergence d'une structure sera simultanée avec une autre, chacune exploitant leur interrelations et environnements générés conditionnés par le système de limites. Nous explorerons ces idées pour le cas des systèmes territoriaux par la suite.
}

\bpar{
This introductive chapter aims to set up the thematic scene, the geographical context in which further developments will root. It is not supposed to be understood as an exhaustive literature review nor the fundamental theoretical basement of our work (the first will be an object of chapter~\ref{ch:quantepistemo} whereas the second will be earlier tackled in chapter~\ref{ch:theory}), but more as narration aimed to introduce typical objects and views and construct naturally research questions.
}{
Ce chapitre introductif est destiné à poser le cadre thématique, le contexte géographique sur lesquels les développements suivants se baseront. Il n'est pas supposé être compris comme une revue de littérature exhaustive ni comme les fondations théoriques fondamentales de notre travail (le premier point étant l'objet du chapitre~\ref{ch:quantepistemo} tandis que le second sera traité plus tôt dans le chapitre~\ref{ch:theory}), mais plutôt comme une construction narrative ayant pour but d'introduire nos objets et positions d'étude, afin de construire naturellement des questions de recherche précises.
}






%-------------------------------

\newpage

\section[Territories and Networks]{Territories and Networks}
%\bsection{Territories and Networks}{Réseaux et Territoires}

%\bsubsection{Territories and Networks : There and Back Again}{Une circularité naturelle}
\subsection{Territories and Networks : There and Back Again}

\paragraph{Human Territories}
%\paragraph{Territorialité Humaine}


\bpar{
The notion of territory can be taken as a basis to explore the scope of geographical objects we will study. In Ecology, a territory corresponds to a spatial extent occupied by a group of agents or more generally an ecosystem. \emph{Human Territories} are far more complex in the sense of semiotic representations of these that are a central part in the emergence of societies. For \noun{Raffestin} in~\cite{raffestin1988reperes}, the so-called \emph{Human Territoriality} is the ``conjonction of a territorial process with an informational process'', what means that physical occupation and exploitation of space by human societies is not dissociable from the representations (cognitive and material) of these territorial processes, driving in return its further evolutions. In other words, as soon as social constructions are assumed in the constitution of human settlements, concrete and abstract social structures will play a role in the evolution of the territorial system, through e.g. propagation of information and representations, political processes, conjonction or disjonction between lived and perceived territory. Although this approach does not explicitly give the condition for the emergence of a seminal system of aggregated settlements (i.e. the emergence of cities), it insists on the role of these that become places of power and of creation of wealth through exchange. But the city has no existence without its hinterland and the territorial system can not be summarized by its cities as a system of cities. There is however compatibility on this subsystem between \noun{Raffestin} approach to territories and \noun{Pumain}'s evolutive theory of urban systems~\cite{pumain2010theorie}, in which cities are viewed as an auto-organized complex dynamical systems, and act as mediators of social changes : for example, cycles of innovation occur within cities and propagate between them. Cities are thus competitive agents that co-evolve (in the sense given before). The territorial system can be understood as a spatially organized social structure, including its concrete and abstract artifacts. A imaginary free-of-man spatial extent with potential ressources will not be a territory if not inhabited, imagined, lived, and exploited, even if the same ressources would be part of the corresponding habited territorial system. Indeed, what is considered as a ressource (natural or artificial) will depend on the corresponding society (e.g. of its practices and technological potentialities). A crucial aspect of human settlements that were studied in geography for a long time, and that relate with the previous notion of territory, are \emph{networks}. Let see how we can switch from one to the other and how their definition may be indissociable.
}{
Une entrée possible dans l'ensemble des objets géographiques que nous proposons d'étudier est la notion de territoire. En Ecologie, un territoire correspond à l'étendue spatiale occupée par un groupe d'agent ou plus généralement un écosystème. Les \emph{Territoires Humains} sont extrêmement plus complexes de par l'importance de leur représentations sémiotiques, qui jouent un rôle significatifs dans l'émergence des constructions sociétales. Selon \noun{Raffestin} dans~\cite{raffestin1988reperes}, la \emph{Territorialité Humaine} est ``la conjonction d'un processus territorial avec un processus informationnel'', ce qui implique que l'occupation physique et l'exploitation de l'espace par les sociétés humaines n'est pas dissociable des représentations (cognitives et matérielles) de ces processus territoriaux, qui influent en retour leur évolution. En d'autres termes, à partir de l'instant où les constructions sociales déterminent la constitution des établissements humains, les structures sociales abstraites et concrètes joueront un role dans l'évolution des systèmes territoriaux, par exemple à travers la propagation d'informations et de représentations, par des processus politiques, ou encore par la correspondance effective entre territoire vécu et territoire perçu. Bien que cette approche ne donne pas de conditions explicites pour l'émergence d'un système séminal d'établissements agrégés (c'est à dire l'émergence des villes), elle insiste sur leur role comme lieu de pouvoir et de création de richesse au travers des échanges. Mais la ville n'a pas d'existence sans son hinterland et le système territorial peut difficilement être résumé par ses villes, comme un système de villes. En se restreignant à ce sous-système, il y a toutefois compatibilité entre la théorie de territoires de \noun{Raffestin} et la théorie évolutive des villes de \noun{Pumain}~\cite{pumain2010theorie}, qui interprète les villes comme des systèmes complexes dynamiques auto-organisés, qui agissent comme des médiateurs du changement social : par exemple, les cycles d'innovation s'initialisent au sein des villes et se propagent entre elles. Les villes sont ainsi des agents compétitifs qui co-évoluent (au sens donné précédemment). Le système territorial peut ainsi être compris comme une structure sociale organisée dans l'espace, qui comprend ses artefacts concrets et abstraits. Une étendue spatiale imaginaire avec des ressources potentielles qui n'aurait jamais connu de contact avec l'humain ne pourra pas être un territoire si elle n'est pas habitée, imaginée, vécue, exploitée, même si ces ressources pourraient être potentiellement exploitée le cas échéant. En effet, ce qui est considéré comme une ressource (naturelle ou artificielle) dépendra de la société (par exemple de ses pratiques et de ses capacité technologiques). Un aspect central des établissements humains qui a une longue tradition d'étude en géographie, et qui est directement relié à la notion de territoire, est celui des \emph{réseaux}. Nous allons voir comment le passage de l'un à l'autre est inévitable et leur définition indissociable.
}


\paragraph{A Territorial Theory of Networks}


\bpar{
We paraphrase \noun{Dupuy} in~\cite{dupuy1987vers} when he proposes elements for ``a territorial theory of networks'' based on the concrete case of Urban Transportation Networks. This theory sees \emph{real networks} (i.e. concrete networks, including transportation networks) as the materialization of \emph{virtual networks}. More precisely, a territory is characterized by strong spatio-temporal discontinuities induced by the non-uniform distribution of agents and ressources. These discontinuities naturally induce a network of ``transactional projects'' that can be understood as potential interactions between elements of the territorial system (agents and/or ressources). For example today, people need to access the ressource of employments, economic exchanges operate between specialized production territories. At any time period, potential interactions existed\footnote{even when nomadism was still the rule, spatially dynamic networks of potential interactions necessarily existed, but should have less chance to materialize into concrete routes.% bib on that ?
}. The potential interaction network is concretized as offer adapts to demand, and results of the combination of economic and geographical constraints with demand patterns, in a non-linear way through agents designed as \emph{operators}. This process is not immediate, leading to strong non-stationarity and path-dependance effects : the extension of an existing network will depend on previous configuration, and depending on involved time scales, the logic and even the nature of operators may have evolved. \noun{Raffestin} points out in his preface of~\cite{offner1996reseaux} that a geographical theory articulating space, network and territories had never been consistently formulated. It appears to still be the case today, but the theory developed just before is a good candidate, even if it stays at a conceptual level. The presence of a human territory necessarily imply the presence of abstract interaction networks and concrete networks used for transportation of people and ressources (including communication networks as information is a crucial ressource). Depending on regime in which the considered system is, the respective role of different networks may be radically different. Following \noun{Duranton} in \cite{duranton1999distance}, pre-industrial cities were limited in growth because of limitations of transportation networks. Technological progresses have lead to the end of these limitations and the preponderance of land markets in shaping cities (and thus a role of transportation network as shaping prices through accessibility), and recently to the rising importance of telecommunication networks that induce a ``tyranny of proximity'' as physical presence is not replaceable by virtual communication. This territorial approach to networks seems natural in geography, since networks are studied conjointly with geographical objects with an underlying theory, in opposition to network science that studies brutally spatial networks with few thematic background~\cite{ducruet2014spatial}.
}{
Nous paraphrasons \noun{Dupuy} dans~\cite{dupuy1987vers} lorsqu'il propose des éléments pour une ``théorie territoriale des réseaux'' basée sur le cas concret d'un réseau de transport urbain. Cette théorie présente les \emph{réseaux réels} (i.e. les réseaux concrets % TODO différent de réseaux matériels ?
, incluant les réseaux de transport) comme la matérialisation de \emph{réseaux virtuels}. Plus précisément, un territoire est caractérisé par de fortes discontinuités spatio-temporelles induites par la distribution non-uniforme des agents et des ressources. Ces discontinuités induisent naturellement un réseau de ``projets transactionnels'' qui peuvent être compris comme des interactions potentielles entre les éléments du système territorial % TODO remarque : cf Chenyi modèles de potentiels -> pertinence de cette approche, même au regard du modèle macro ?
(agents et/ou ressources). Par exemple, de nos jours les actifs se doivent d'accéder à la ressource qu'est l'emploi, et des échanges économiques s'effectuent entre les différents territoires spécialisés dans les productions de différents types. En tout temps des interactions potentielles ont existé\footnote{même quand le nomadisme devait encore être la règle, des réseaux d'interactions potentielles dynamiques dans l'espace ont du exister, mais devaient avoir moins de chance de se matérialiser en des routes matérielles.} Le réseau d'interaction potentiel est concrétisé quand l'offre s'adapte à la demande, et résulte en la combinaison de contraintes économiques et géographiques avec les motifs de demande, de manière non-linéaire via des agents qu'on peut désigner comme \emph{opérateurs}. Un tel processus est loin d'être immédiat, et conduit à de forts effets de non-stationarité et de dépendance au chemin : l'extension d'un réseau existant dépendra de la configuration précédente, et selon les échelles de temps impliquées, la logique et même la nature des opérateurs peut avoir évolué. \noun{Raffestin} souligne dans sa préface de~\cite{offner1996reseaux} qu'une théorie géographique articulant espaces, réseaux et territoires n'a jamais été formulée de manière cohérente. Il semble que c'est toujours le cas aujourd'hui, même si la théorie évoquée ci-dessus semble être un bon candidat bien qu'elle reste à un niveau conceptuel. La présence d'un territoire humain implique nécessairement la présence de réseaux d'interactions abstraites et de réseaux concrets utilisés pour transporter les individus et les ressources (incluant les réseaux de communication puisque l'information est une ressource essentielle). Selon le régime dans lequel le système considéré se trouve, le rôle respectif du réseau peut être radicalement différent. Selon \noun{Duranton}~\cite{duranton1999distance}, les villes pré-industrielles étaient limitées en croissance de par les limitations des réseaux de transport. Les progrès technologiques ont permis de les surmonter et à mené à la prépondérance du marché foncier dans la formation des villes (et par conséquent un rôle des réseaux de transport qui déterminent les prix par l'accessibilité), et plus récemment à une importance croissante des réseaux de télécommunication ce qui a induit une ``tyrannie de la proximité'' puisque la présence physique n'est pas remplaçable par une communication virtuelle. Cette approche territoriale des réseaux semble naturelle en géographie, puisque les réseaux sont étudiés conjointement avec des objets géographiques auxquels est associée une théorie, en opposition à la science des réseaux qui étudie brutalement les réseaux spatiaux avec peu de fond thématique~\cite{ducruet2014spatial}.
}

\paragraph{Networks shaping territories ?}

% how do network shape territories : boundaries, scales, etc.
% example : \cite{l2012ville} bahn-ville, volontary coevol ? // idem villes nouvelles


\bpar{
However networks are not only a material manifestation of territorial processes, but play their part in these processes as they evolution may shape territories in return. In the case of \emph{technical networks}, an other designation of real networks given in~\cite{offner1996reseaux}, many examples of such feedbacks can be found : the interconnectivity of transportation networks allows multi-scalar mobility patterns, thus shaping the lived territory. At a smaller scale, changes in accessibility may result in an adaptation of a functional urban space. Here emerges again an intrinsic difficulty : it is far from evident to attribute territorial mutations to some network evolutions and reciprocally materialization of a network to precise territorial dynamics. Coming back to Diderot should help, in the sense that one must not consider network nor territories as independent systems that would have causal relationships but as strongly coupled components of a larger system. The confusion on possible simple causal relationships has fed a scientific debate that is still active. Methodologies to identify so-called \emph{structural effects} of transportation networks were proposed by planners in the seventies~\cite{bonnafous1974detection,bonnafous1974methodologies}. It took some time for a critical positioning on unreasoned and decontextualized use of these methods by planners and politics generally to technocratically justify transportation projects, that was first done by \noun{Offner} in~\cite{offner1993effets}. Recently the special issue~\cite{espacegeo2014effets} on that debate recalled that on the one hand misconceptions and misuses were still greatly present in operational and planning milieus as~\cite{crozet:halshs-01094554} confirmed, and on the other hand that a lot of scientific progresses still need to be made to understand relations between networks and territories as \noun{Pumain} highlights that recent works gave evidence of systematic effects on very long time scales (as e.g. the work of \noun{Bretagnolle} on railway evolution, that shows a kind of structural effect in the necessity of connectivity to the network for cities to ``stay in the game'', but that is not fully causal as not sufficient). At a macroscopic level typical patterns of interaction emerge, but microscopic trajectories of the system are essentially chaotic : the understanding of coupled dynamics strongly depends on the scale considered. At a small scale it seems indeed impossible to show systematic behavior, as \noun{Offner} pointed out. For example, on comparable French mountain territories, \cite{berne2008ouverture} shows that reactions to a same context of evolution of the transportation network can lead to very different reactions of territories, some finding a huge benefit in the new connectivity, whereas others become more closed. These potential retroactions of networks on territories does not necessarily act on concrete components : \noun{Claval} shows in~\cite{claval1987reseaux} that transportation and communication networks contribute to the collective representation of territories by acting on territorial belonging feeling.
}{
Cependant les réseaux ne sont pas seulement une manifestation matérielle de processus territoriaux, mais jouent également leur rôle dans ces processus comme leur évolution peut influencer l'évolution des territoires en retour. Dans le cas des \emph{réseaux techniques}, une autre désignation des réseaux réels donnée dans~\cite{offner1996reseaux}, de nombreux exemples de tels retroactions peuvent être mis en évidence : l'interconnexion des réseaux de transport permet des motifs de mobilité multi-échelles, formant ainsi le territoire vécu. A une plus petite échelle, des changements de l'accessibilité peuvent induire l'adaptation d'un espace fonctionnel urbain. Il emerge alors une difficulté intrinsèque : il est loin d'évident d'attribuer des mutations territoriales à une évolution du réseau and réciproquement la matérialisation d'un réseau à des dynamiques territoriales précises. Revenir à la citation de Diderot devrait aider à ce point, au sens où il ne faut pas considérer le réseau ni les territoires comme des systèmes indépendants qui s'influenceraient mutuellement par des relations causales, mais comme des composantes fortement couplées d'un système plus large. La confusion autour de possibles relations causales simples a nourri un débat scientifique encore actif aujourd'hui. Les méthodologies pour identifier ce qui est nommé \emph{effets structurants} des réseaux de transport ont été proposées par les planificateurs dans les années 1970~\cite{bonnafous1974detection,bonnafous1974methodologies}. Il aura fallu un certain temps pour un positionement critique sur l'usage non raisonné et decontextualisé de ces méthodes par les planificateurs et les politiques généralement pour justifier technocratiquement des projets de transports. Cela a été fait en premier par \noun{Offner} dans~\cite{offner1993effets}. Récemment un édition spéciale du même journal sur ce débat~\cite{espacegeo2014effets} a rappelé d'une part que les mauvaises interprétations et les mauvais usages étaient encore largement présent aujourd'hui dans les milieux opérationnels de la planification comme~\cite{crozet:halshs-01094554} confirme, et d'autre part qu'il faudrait encore une certaine quantité de progrès scientifique pour compréendre en profondeur les relations entre réseaux et territoires. \noun{Pumain} souligne que des travaux récents ont révélé des effets systématiques sur de très longues échelles temporelles (comme e.g. le travail de \noun{Bretagnolle} sur l'évolution des chemins de fer, qui montre une sorte d'effet structurel sur la nécessité de connexion au réseau des villes, afin de rester actives, mais qui n'est ni suffisant ni totalement causal). A un niveau macroscopique des motifs typiques d'interaction émergent, mais les trajectoires microscopiques du systèmes sont essentiellement chaotiques : la compréhension des dynamiques couplées dépend fortement de l'échelle considérée. A une petite échelle il est peu raisonnable de vouloir montrer des comportement systématiques, comme le rappelle \noun{Offner}. Par exemple, sur des territoires de montagne français comparables, \cite{berne2008ouverture} montre que les réactions à un même contexte d'évolution du réseau de transport peut mener à des réactions territoriales très diverses, certains trouvant de forts bénéfices par la nouvelle connectivité, d'autres au contraire devenant plus fermés. Ces retroactions potentielles des réseaux sur les territoires n'agit pas nécessairement sur des composantes concretes : \noun{Claval} montre dans~\cite{claval1987reseaux} que les réseaux de transport et de communication contribuent à la représentation collective d'un territoire en agissant sur un sentiment d'appartenance. % TODO interesting, put that into perspective // DPR ?
}



\paragraph{Territorial Systems}


\bpar{
This detour from territories, to networks and back again, allows us to give a preliminary definition of a territorial system that will be the basis of our following theoretical considerations. As we emphasized the role of networks, the definition takes it into account.
}{
Ce voyage des territoires aux réseaux, et retour, nous permet d'esquisser une définition préliminaire d'un système territorial sur laquelle se basera les considérations théoriques suivantes. Comme nous avons mis en exergue le rôle des réseaux, la définition se doit de les prendre en compte.
}



\bigskip


\bpar{
\textbf{Preliminary Definition.} \textit{A territorial system is a human territory to which both interaction and real networks can be associated. Real networks are a component of the system, involved in evolution processes, through multiples feedbacks with other components at various spatial and temporal scales.}
}{
\textbf{Définition provisoire.} \textit{Un Système Territorial est un territoire humain auquel peuvent être associés à la fois un réseau d'interactions et un réseau réel. Les réseaux réels sont une composante à part entière du système, jouant dans les processus d'évolution, au travers de multiples retroactions avec les autres composantes à plusieurs échelles spatiales et temporelles.}
}

\bigskip



\bpar{
This reading of territorial systems is conditional to the existence of networks and may discard some human territories, but it is a deliberate choice that we justify by previous considerations, and that drives our subject towards the study of interactions between networks and territories.
}{
Cette lecture des systèmes territoriaux est conditionnée à l'existence des réseaux et pourrait écarter certains territoires humains, mais il s'agit d'un choix délibéré justifié par les considérations précédentes, et qui précise notre sujet vers l'étude des interactions entre réseaux et territoires.
}



\subsection{Transportation Networks}


\paragraph{The particularity of transportation networks}



\bpar{
Already evoked in relation to the question of structural effects of networks, transportation networks play a determining role in the evolution of territories. Although other types of networks are also strongly involved in the evolution of territorial systems (see e.g. the discussions of impacts of communication networks on economic activities), transportation networks shape many other networks (logistics, commercial exchanges, social concrete interactions to give a few) and are prominent in territorial evolution patterns, especially in our recent societies that has become dependent of transportation networks~\cite{bavoux2005geographie}. The development of French High Speed Rail network is a good illustration of the impact of transportation networks on territorial development policies. Presented as a new era of railway transportation, a top-down planning of totally novel lines was introduced as central for developments~\cite{zembri1997fondements}. The lack of integration of these new networks with existing ones and with local territories is now observed as a structural weakness and negative impacts on some territories have been shown~\cite{zembri2008contribution}. A review done in~\cite{bazin2011grande} confirms that no general conclusions on local effects of High Speed lines connection can be drawn although it keeps a strong place in imaginaries. These are examples of how transportation networks have both direct and indirect impacts on territorial dynamics. Integrated planning, in the sense of a joint planning of transportation infrastructures and urban development, considers the network as a determining component of the territorial system. Parisian \emph{Villes Nouvelles} are such a case, that witnesses of the complexity of such planning actions that generally do not lead to the desired effect~\cite{es119}. Recent projects as~\cite{l2012ville} have try to implement similar ideas but we have now not enough temporal scope to judge their success in effectively producing an integrated territory. Transportation networks are anyway at the center of these approaches of urban territories. We will focus in our work on transportation networks for the various reasons given here.
}{
Déjà évoqués dans le cas des effets structurants des réseaux, les réseaux de transports jouent un rôle déterminant dans l'évolution des territoires. Même si d'autres types de réseaux sont également fortement impliqués dans l'évolution des systèmes territoriaux (voir e.g. les débats sur l'impact des réseaux de communication sur la localisation des activités économiques), les réseaux de transport conditionnent d'autres types de réseaux (logistique, échanges commerciaux, interactions sociales concrètes pour donner quelques exemples) and semblent dominer dans les motifs d'évolution territoriale, en particulier dans nos sociétés contemporaines qui sont devenues dépendantes des réseaux de transport~\cite{bavoux2005geographie}. Le développement du réseau français à grande vitesse est une illustration pertinente de l'impact des réseaux de transport sur les politiques de développement territorial. Présenté comme une nouvelle ère de transport sur rail, une planification par le haut de lignes totalement nouvelles a été présenté comme central pour le développement~\cite{zembri1997fondements}. Le manque d'intégration de ces nouveaux réseaux avec l'existant et avec les territoires locaux est à présent observé comme une faiblesse structurelle et des impacts négatifs sur certains territoires ont été prouvés~\cite{zembri2008contribution}. Une revue faite dans~\cite{bazin2011grande} confirme qu'aucune conclusion générale sur des effets locaux d'une connection à une ligne à grande vitesse ne peut être tirée, bien que ce sésame garde une place conséquente dans les imaginaires des élus. Ces exemples illustrent comment les réseaux de transport peuvent avoir des effets à la fois directs et indirects sur les dynamiques territoriales. La planification intégrée, au sens d'une planification coordonnée entre les infrastructures de transport et le développement urbain, considère le réseau comme une composante déterminante du système territorial. % TODO : note pub TOD in Zhuhai near BeiZhan -- develop on that // in the fieldwork report
Les Villes Nouvelles parisiennes sont un tel cas qui témoigne de la complexité de ces actions de planification qui le plus souvent ne mène pas au effets initialement désirés~\cite{es119}. Des projets récents comme~\cite{l2012ville} ont tenté d'implémenter des idées similaires, mais il manque pour l'instant de recul pour juger de leur succès à produire un territoire effectivement intégré. Les réseaux de transports sont dans tous les cas au centre de ces approches des territoires urbains. Nous nous concentrerons par la suite sur les réseaux de transport pour toutes ces raisons évoquées ici.
}



\paragraph{Deconstructing Accessibility}

% critic of accessibility as a planning tool : danger of not taking into account socio-eco dynamics and coupled dynamics (coevol) - cit Hadri mobility as a constructed notion.


\bpar{
The notion of accessibility comes rapidly when considering transportation networks. Based on the possibility to access a place through a transportation network (including transportation speed, difficulty of travel), it is generally described as a potential of spatial interaction\footnote{and often generalized as \emph{functional accessibility}, for example employments accessible for actives at a location. Spatial interaction potentials ruling gravity law can also been understood this way.}~\cite{bavoux2005geographie}. This object is often used as a planning tool or as an explicative variable of agents localisation for example. One has to be however careful on its unconditional use. More precisely, it may be a construction that misses a consistent part of territorial dynamics. The mystification of the notion of \emph{mobility} was shown by \noun{Commenges} in~\cite{commenges:tel-00923682}, which proved than most of debates on modeling mobility and corresponding notions were mostly made-of by transportation administrators of \emph{Corps des Ponts} who roughly imported ideas from the United States without adaptation and reflexion fit to the totally different French context. Accessibility may be such a social construct and have no theoretical root since it is mostly a modeling and planning tool. Recent debates on the planification of \emph{Grand Paris Express}~\cite{confMangin}, a totally novel metropolitan transportation infrastructure planned to be built in the next twenty years, have revealed the opposition between a vision of accessibility as a right for disadvantaged territories against accessibility as a driver of economic development for already dynamic areas, both being difficultly compatible since corresponding to very different transportation corridors. Such operational issues confirm the complexity of the role of transportation networks in the dynamics of territorial systems, and we shall give in our work elements of response to a definition of accessibility that would integrate intrinsic territorial dynamics.
}{
La notion d'accessibilité surgit rapidement lorsqu'on s'intéresse aux réseaux de transport. Basée sur la possibilité d'accéder un lieu par un réseau de transport (pouvant prendre en compte la vitesse, la difficulté de se déplacer), elle est généralement définie comme un potentiel d'interaction spatiale\footnote{et souvent généralisée comme une \emph{accessibilité fonctionnelle}, par exemple les emplois accessibles aux actifs d'un lieu. Les potentiels d'interaction spatiaux s'exprimant dans les lois de gravité peuvent aussi être compris de cette façon.}~\cite{bavoux2005geographie}. Cet objet est souvent utilisé comme un outil de planification ou comme une variable explicative de localisation des agents par exemple. Il faut cependant rester prudent sur son usage inconditionnel. Plus précisément, il peut s'agir d'une construction qui ignore une partie conséquente des dynamiques territoriales. La mystification de la notion de \emph{mobilité} a été montrée par \noun{Commenges} dans~\cite{commenges:tel-00923682}, qui révèle que la majorité des débats sur la modélisation de la mobilité et les notions correspondantes était majoritairement construites de manière ad-hoc par les administrateurs de transports issus du \emph{Corps des Ponts} qui importaient brutalement les outils et méthodes des Etats-Unis sans adaptation ni reflexion adaptée au contexte français. L'accessibilité pourrait de même être une construction sociale et n'avoir que peu de fondement théorique, puisqu'il s'agit en grande partie d'un outil de modélisation et de planning. Les débats récents sur la planification du \emph{Grand Paris Express}~\cite{confMangin}, cette nouvelle infrastructure de transport métropolitaine
}

\paragraph{Scales and Hierarchies}


% \cite{10.1371/journal.pone.0102007}
% \cite{Tsekeris20131} : congestion related to land-use


\bpar{
An incontournable aspect of transportation networks that we will need to take into account in further developments is hierarchy. Transportation networks are by essence hierarchical, depending on scales they are embedded in. \cite{10.1371/journal.pone.0102007} showed empirical scaling properties for public transportation networks for a consequent number of metropolitan areas across the world, and scaling laws reveal the presence of hierarchy within a system, as for size hierarchy for system of cities expressed by Zipf's law~\cite{nitsch2005zipf} or other urban scaling laws\cite{2013arXiv1301.1674A,2015arXiv151000902B}. Transportation network topology has been shown to exhibit such scaling also for the distribution of its local measures such as centrality~\cite{samaniego2008cities}. Hierarchy seems to play a particular role on interaction processes, as \noun{Bretagnolle}~\cite{bretagnolle:tel-00459720} highlighted an increasing correlation between urban hierarchy and network hierarchy for French railway network, marker of positive feedbacks between urban rank and network centralities. Different regimes in space and times were identified : for French railway network evolution e.g., a first phase of adaptation of the network to the existing urban configuration was followed by a phase of co-evolution i.e. in the sense that causal relations become difficult to identify. Railway evolution in the United States followed a different pattern, without hierarchical diffusion, shaping locally urban growth. It emphasizes the presence of path-dependance for trajectories of urban systems : the presence in France of a previous city system and network (postal roads) strongly shaped railway development, whereas its absence in the US lead to completely different dynamics. An open question is if generic processes underlie both evolutions, each being different realizations with different initial conditions and different meta-parameters (different \emph{regimes} in the sense of settlement systems transitions introduced in the current ANR Research project TransMonDyn, as a transition can be understood as a change of stationarity for meta-parameters of a general dynamic). In terms of dynamical systems formulation, it is equivalent to ask if dynamics of attractors (long time scale components) obey similar equations as the position and nature of attractors for a stochastic dynamical system give its current regime, in particular if it is in a divergent state (positive local Liapounov exponent) or is converging towards stable mechanisms~\cite{sanders1992systeme}. To answer this question together with a disentangling of co-evolution processes for that regime, \cite{bretagnolle:tel-00459720} proposes modeling as a constructive element of answer. We will see in next section how modeling can bring knowledge about territorial processes.
}{

}


\paragraph{Interactions between transportation networks and territory}

At this state of progress, we have naturally identified a research subject that seems to take a significant place in the complexity of territorial systems, that is the study of interactions between transportation networks and territories. In the frame of our preliminary definition of a territorial system, this question can be reformulated as the study of networked territorial systems with an emphasize on the role of transportation networks in system evolution processes.




%----------------------------------------------------------------------------------------

\newpage

\section{Modeling Interactions}


\subsection{Modeling in Quantitative Geography}

% brief reference to the history of TQG ; history of modeling.
%  note : history of future of TQG, London september 2016



\bpar{
Modeling in Theoretical and Quantitative Geography (TQG), and more generally in Social Science, has a long history on which we can not go further than a general context. \noun{Cuyala} does in~\cite{cuyala2014analyse} an analysis of the spatio-temporal development of French speaking TQG movement and underlines the emergence of the discipline as the combination between quantitative analysis (e.g. spatial analysis or modeling and simulation practices) and theoretical constructions, an integration of both allowing the construction of theories from empirical stylized facts that yield theoretical hypothesis to be tested on empirical data. These approach were born under the influence of the \emph{new geography} in Anglo-saxon countries and Sweden. A broad history of the genesis of models of simulation in geography is done by \noun{Rey} in~\cite{rey2015plateforme} with a particular emphasis on the notion of validation of models. The use of computation for simulation of models is anterior to the introduction of paradigms of complexity, coming back to \noun{H{\"a}gerstrand} and \noun{Forrester}, pioneers of spatial economic models inspired by Cybernetics. With the increase of computational possibilities epistemological transformations have also occurred, with the apparition of explicative models as experimental tools. \noun{Rey} compares the dynamism of seventies when computation centers were opened to geographers to the democratization of High Performance Computing (transparent grid computing, see~\cite{schmitt2014half} for an exemple of the possibilities offered in terms of model validation and calibration, decreasing the computational time from 30 years to one week), that is also accompanied by an evolution of modeling practices~\cite{banos2013pour} and techniques~\cite{10.1371/journal.pone.0138212}. Modeling (in particular computational models of simulation) is seen by many as a fundamental building brick of knowledge : \cite{livet2010} recalls the combination of empirical, conceptual (theoretical) and modeling domains with constructive feedbacks between each. A model can be an exploration tool to test assumptions, an empirical tool to validate a theory against datasets, an explicative tool to reveal causalities (and thus internal processes of a system), a constructive tool to iteratively build a theory with an iterative construction of an associated model. These are example among others : \noun{Varenne} proposes in~\cite{varenne2010simulations} a refined classifications of diverse functions of a model. We will consider modeling as a fundamental instrument of knowledge on processes within complex adaptive systems, as already evoked, and restraining again our question, will focus on \emph{models involving interactions between transportation networks and territories}.
}{
La modélisation en Géographie Théorique et Quantitative (TQG), et plus généralement en Sciences Sociales, a une longue histoire dont nous ne pourrons que brosser un bref portrait ici. \noun{Cuyala} procède dans~\cite{cuyala2014analyse} à une analyse spatio-temporelle du mouvement de la Géographie Théorique et Quantitative en langue française et souligne l'émergence de la discipline comme une combinaison d'analyses quantitatives (e.g. analyse spatiale et pratiques de modélisation et de simulation) et de construction théoriques. L'intégration de ces deux composantes permet la construction de théories à partir de faits stylisés empiriques, qui produisent à leur tour des hypothèses théoriques pouvant être testées sur les données empiriques. Cette approche est née sous l'influence de la \emph{New Geography} dans les pays Anglo-saxons et en Suède. Une histoire étendue de la genèse des modèles de simulation en géographie est faite par \noun{Rey} dans~\cite{rey2015plateforme} avec une attention particulière pour la notion de validation de modèles. L'utilisation de ressources de calcul pour la simulation de modèles est antérieur à l'introduction des paradigmes de la complexité, remontant à \noun{H{\"a}gerstrand} et \noun{Forrester}, pionniers des modèles d'économie spatiale inspirés par la cybernétique. Avec l'augmentation des potentialités de calcul, des transformations épistémologiques ont également suivi, avec l'apparition de models explicatifs comme outils expérimentaux. \noun{Rey} compare le dynamisme des années soixante-dix quand les centres de calcul furent ouverts aux géographes à la démocratisation actuelle du Calcul Haute Performance (calcul sur grille à l'utilisation transparente, voir~\cite{schmitt2014half} pour un exemple des possibilités offertes en terme de calibration et de validation de modèle, réduisant le temps de calcul nécessaire de 30 ans à une semaine), qui est également accompagnée par une évolution des pratiques~\cite{banos2013pour} et techniques~\cite{10.1371/journal.pone.0138212} de modélisation. La modélisation, et en particulier les modèles de simulation, est vue par beaucoup comme une brique fondamentale de la connaissance : \cite{livet2010} rappelle la combinaison des domaines empirique, conceptuel (théorique) et de la modélisation, avec des retroactions constructives entre chaque. Une modèle peut être un outil d'exploration pour tester des hypothèses, un outil empirique pour valider une théorie sur des jeux de données, un outil explicatif pour révéler des causalités et ainsi des processus internes au système, un outil constructif pour construire itérativement une théorie conjointement avec celle des modèles associés. Ce sont des exemples de fonctions parmi d'autres : Varenne donne dans~\cite{varenne2010simulations} une classification raffinée des diverses fonctions d'un modèle. Nous considérons la modélisation comme un instrument fondamental de connaissance des processus au sein de systèmes complexes adaptatifs, et précisons encore notre question de recherche, qui s'intéressera aux \emph{modèles impliquant des interactions réseaux et territoires}.
}


\subsection{Modeling Territories and Networks}

% here overview of different approaches
% Q : do it here, not during quant epistemo part ?


\bpar{
Concerning our precise question of interactions between transportation networks and territories, we propose an overview of existing approaches. Following~\cite{bretagnolle2002time}, the ``\textit{thoughts of specialists in planning aimed to give definitions of city systems, since 1830, are closely linked to the historical transformations of communication networks}''. It implies that ontologies and corresponding models addressed by geographers and planners are closely linked to their current historical preoccupations, thus necessarily limited in scope and purpose. In a perspectivist vision of science~\cite{giere2010scientific} such boundaries are the essence of the scientific entreprise, and as we will argue in chapter~\ref{ch:theory} their combination and coupling in the case of models is a source of knowledge.
}{

}



\subsubsection{Land-Use Transportation Interaction Models}



\bpar{
A subsequent bunch of literature in modeling interaction between networks and territories can be found in the field of planning, with the so-called \emph{Land-use Transportation Interaction Models}. These works are difficult to be precisely bounded as they may be influenced by various disciplines. For example, from the point of view of Urban Economics, propositions for integrated models have existed for a relatively long term~\cite{putman1975urban}. The variety of possible models has lead to operational comparisons~\cite{paulley1991overview,wegener1991one}. More recently, the respective advantages of static and dynamic modeling was investigated in~\cite{kryvobokov2013comparison}. Generally these type of models operate at relatively small temporal and spatial scales. \cite{wegener2004land} reviewed state of the art in empirical and modeling studies on interactions between land-use and transportation. It is positioned in economic, planning and sociological theoretical contexts, and is relatively far from our geographical approach aiming to also understand long-time processes. Seventeen models are compared and classified, none of which implements actually network endogenous evolution on the relatively small time scales of simulation. A complementary review done in \cite{chang2006models} broadens the scope with inclusion of more general classes of models, such as spatial interaction models (including traffic assignment and four steps models), operational research planning models (optimal localisations), micro-based random utility models, and urban market models. These techniques operate also at small scales and consider at most land-use evolution. \cite{iacono2008models} covers a similar scope with a further emphasis on cellular automata models of land-use change and agent-based models. These type of models are still largely developed and used today, as for example \cite{delons:hal-00319087} which is used for Parisian metropolitan region. The short-term range of application and their operational character makes them useful for planning, what is far from our preoccupation to obtain explicative models for geographical processes. 
}{
Un partie importante de la littérature proposant des modélisations des interactions entre réseaux et territoires se trouve dans le domaine de la planification urbaine, avec les \emph{modèles d'interaction entre usage du sol et transport} (\emph{LUTI}). Ces travaux peuvent être difficiles à cerner car liés à différentes disciplines. Par exemple, du point de vue de l'Economie Urbaine, les propositions de modèle intégrés existent depuis un certain temps~\cite{putman1975urban}. La variété des modèles existants a conduit à des comparaisons opérationnelles~\cite{paulley1991overview,wegener1991one}. Plus récemment, les avantages respectifs des approches statiques et dynamiques a été étudié par~\cite{kryvobokov2013comparison}. Dans tous les cas, ce type de modèle opère généralement à des échelles temporelles et spatiales relativement faibles.  \cite{wegener2004land} donne un état de l'art des études empiriques et de modélisation sur ce type d'approche des interactions entre usage du sol et transport. Le positionnement théorique est plutôt proche des disciplines de l'Economie, de la Planification et de la Sociologie, et relativement de nos raisonnements géographiques qui se veulent de comprendre également des processus sur le temps long. Pas moins de dix-sept modèles sont comparés et classifiés, parmi lesquels aucun n'inclut une évolution endogène du réseau de transport sur les échelles de temps relativement petites des simulations. Une revue complémentaire est faite par~\cite{chang2006models}, élargissant le contexte avec l'inclusion de classes plus générales de modèles, comme des modèles d'interactions spatiales (parmi lesquels l'attribution du traffic et les modèles à quatre temps), les modèles de planification basés sur la recherche opérationnelle (optimisation des localisations), les modèles microscopiques d'utilité aléatoire, et les modèles de marché foncier. Toutes ces techniques opèrent également à une petite échelle et considèrent au plus l'évolution de l'usage du sol. \cite{iacono2008models} couvre un horizon similaire avec une emphase supplémentaire sur les modèles à automates cellulaires d'évolution d'usage du sol et les modèles basés agent. Les modèles LUTI sont toujours largement étudiés et appliqués, comme par exemple \cite{delons:hal-00319087} qui est utilisé pour la région métropolitaine parisienne. La courte portée temporelle d'application de ces modèles et leur nature opérationnelle les rend utiles pour la planification, ce qui est assez loin de notre souci d'obtenir des modèles explicatifs de processus géographiques.
}




\subsubsection{Network Growth}

% economic models
%\cite{yerra2005emergence} % : cost-driven model of nw reinforcement (// slime mould)
%\cite{louf2013emergence} % : trade-off between cost and benefits due to flows -> compare with lutecia rules ?
%\cite{xie2009modeling} review of network growth economic appraoches
%\cite{bigotte2010integrated} % : planning network - similar to nw growth ? - hierarchy in cities and nw
 
 % geometric - local optimization
%\cite{barthelemy2008modeling}
%\cite{courtat2011mathematics} % measures and morphogenesis. Vision of morphogenesis as living organism : nuance that in theory.
%\cite{de2007netlogo} % geom rules in Tijuana model
%\cite{rui2013exploring} % local based optimisation morphogenesis model. RQ : quote only physicists work -> justification for extended quantitative epistemology ?
%\cite{yamins2003growing} strange model
 
% biological nws

%\cite{tero2010rules} % physarum : biological nw heuristics
%\cite{tero2006physarum} % potentialities of physarum machines, here for routing.
%\cite{adamatzky2010road} % planning absurdities
%\cite{zhu2013amoeba} % TSP solving : long range correlations.


\bpar{
Network growth can be used to design modeling entreprises that aim to endogenously explain growth of transportation networks, generally from a bottom-up point of view, i.e. by exhibiting local rules that would allow to reproduce network growth over long time scales (generally the road network). Economists have proposed such models : \cite{zhang2007economics} reviews transportation economics literature on network growth within an endogenous growth theory~\cite{aghion1998endogenous}, recalling the three main features studied by economists on that subject that are road pricing, infrastructure investment and ownership regime, and describes an analytical model combining the three.
\cite{xie2009modeling} develops a broad review on network growth modeling extending to other fields : transportation geography early developed empirical-based models but which did concentrate on topology reproduction rather than on mechanisms according to~\cite{xie2009modeling} ; statistical models on case studies provide mitigated conclusions on causal relations between offer and demand ; economists have studied infrastructure provision from both microscopic and macroscopic point of views, generally non-spatial ; network science has provided toy-models of network growth based on structural and topological rules rather on mechanism-based rules. An other approach not mentioned that we will develop further is biologically inspired network design. We first give some example of economic-based and geometrical-based network growth modeling attempts. \cite{yerra2005emergence} shows through a reinforcement economic model including investment rule based on traffic assignment that local rules are enough to make hierarchy of roads emerge for a fixed land-use. A very similar model in~\cite{louf2013emergence} with simpler cost-benefits obtains the same conclusion. Whereas these models based on processes focus on reproducing macroscopic patterns of networks (typically scaling), geometrical optimization models aim to ressemble topologically real networks. \cite{barthelemy2008modeling} proposes a model based on local energy optimization but it stays very abstract and unvalidated. The morphogenesis model given in~\cite{courtat2011mathematics} based on local potential and connectivity rules, even if not calibrated, seems to reproduce more reasonably real street patterns. Very close work is done in~\cite{rui2013exploring}.
Other tentatives \cite{de2007netlogo,yamins2003growing} are closer to procedural modeling~\cite{lechner2004procedural,watson2008procedural} and therefore not of interest in our purpose as they can difficultly be used as explicative models. Finally, an interesting and original approach to network growth are biological networks. These belong to the field of morphogenetic engineering pioneered by \noun{Doursat} that aim to design artificial complex system inspired from natural complex systems and in which a control of emerging properties is possible~\cite{doursat2012morphogenetic}. \emph{Physarum Machines}, that are models of a self-organized mould (slime mould) have been shown to provide efficient bottom-up solution to computationally heavy problems such as routing problems~\cite{tero2006physarum} or NP-complete navigation problems such as the Travelling Salesman Problem~\cite{zhu2013amoeba}. It has been shown to produce networks with Pareto-efficient cost-robustness properties~\cite{tero2010rules}, relatively close in shape to real networks (under certain conditions, see~\cite{adamatzky2010road}). This type of models can be of interest for us since auto-reinforcement mechanisms based on flows are analog to mechanisms of link reinforcement in transportation economics.
}{

}

% note : comparison with Francois model for french railway (nothing published yet)




\subsubsection{Hybrid Modeling}



%\cite{bigotte2010integrated}
%\cite{levinson2007co} : economic model of coevolution. % check timescales if are consistent.
%\cite{levinson2005paving} % markov chain : not really modeling but more statistics.
%\cite{raimbault2014hybrid}



Models of simulation implementing a coupled dynamic between urban growth and transportation network growth are relatively rare, and always rather poor from a theoretical and thematic point of view. A generalization of the geometrical local optimization model described before was developed in~\cite{barthelemy2009co}. % pb of scales, def of coevolution, thematic meaning of assumptions, etc.
As for the road growth model of which it is an extension, no thematic nor theoretical justification of local mechanisms is provided, and the model is furthermore not explored and no geographical knowledge can be drawn from it. \cite{levinson2007co} adopts a more interesting economic approach, similar to a four step model (gravity-based origin-destination flows generation, stochastic user equilibrium traffic assignment) including travel cost and congestion, coupled with a road investment module simulating toll revenues for constructing agents, and a land-use evolution module updating actives and employments through discrete choice modeling. The experiments showed that co-evolving network and land uses lead to positive feedbacks reinforcing hierarchy, but are far from satisfying for two reasons : first network topology does not really evolve as only capacities and flows change within the network, what means that more complex mechanisms on longer time scales are not taken into account, and secondly the conclusions are very limited as model behavior is not known since sensitivity analysis is done on few one-dimensional spaces : exhaustive mechanisms stay thus unrevealed as only particular cases are described in the sensitivity analysis. From an other point of view, \cite{levinson2005paving} is also presented as a model of co-evolution, but corresponds more to coupled statistical analysis as it relies on a Markov-chain predictive model. \cite{rui2011urban} gives a model in which coupling between land-use and network growth is done in a weak paradigm, land-use and accessibility having no feedback on network topology evolution. \cite{achibet2014model} describes a co-evolution model at a very small scale (scale of the building), in which evolution of both network and buildings are ruled by a same agent (influenced differently by network topology and population density) what implies a too strong simplification of underlying processes. Finally, a simple hybrid model explored and applied to a toy planning example in~\cite{raimbault2014hybrid}, relies on urban activities accessibility mechanisms for settlement growth with a network adapting to urban shape. The rules for network growth are too simple to capture processes we are interested in, but the model produces at a small scale a broad range of urban shapes reproducing typical patterns of human settlements.


\subsubsection{Urban Systems Modeling}

% differentiate to other hybrid models : SimpopNet and others ? (if exist ?)

An approach closer to our current questioning is the one of integrated modeling of system of cities. In the continuity of Simpop models for city systems modeling, \noun{Schmitt} described in~\cite{schmitt2014modelisation} the SimpopNet model which aim was precisely to integrate co-evolution processes in system of cities on long time scales, typically rules for hierarchical network development as a function of cities dynamics coupled with city dynamics depending on network topology. Unfortunately the model was not explored nor further studied, and furthermore stayed at a toy-level. \noun{Cottineau} proposed transportation network endogenous growth as the last building bricks of her Marius productions but it stayed at a conceptual construction stage. We shall position more in that stream of research in this thesis.





\provisory{

\subsection{Sketch of a \emph{Modelography}}

An ongoing work is the production of a synthesis of this overview, from a modular modeling point of view, combined with a purpose and scale classification. Already mentioned, modular modeling consists in the integration of heterogeneous processes and implementation of processes in order to extract the set of mechanisms giving the best fit to empirical data~\cite{cottineau2015incremental}. We can thus classify models described here according to their building bricks in terms of processes implemented and thus identify possible coupling potentialities. This work is a preliminary step for the analysis in quantitative epistemology developed in chapter~\ref{ch:quantepistemo}.

}




%-------------------------


\section[Observation Flottante]{De la Recherche Qualitative : une Experience en Observation Flottante}


\bpar{}{
Si le diable est dans les détails, les systèmes de transport entre autres sont l'allégorie de cette adage. Ce que certains appellent détail contient la majorité de l'information pour d'autres. Logiquement enfermés dans une bulle scientifique, malgré toutes les volontés développées en introduction, on tâchera de rester conscient de la nature et la portée de la connaissance produite ici. Ce que nous pourrions appeler détail, lors de l'étude de l'accessibilité d'un réseau de transport par exemple, tel des impressions ressenties par les usagers ou les relations sociales induites par les situations découlant des dynamiques du systèmes, seront le centre du questionnement pour un anthropologue ou sociologue. Une telle connaissance, qui trouverait certainement une place dans nos problématiques, est hors de notre portée de par l'absence de \emph{terrain} de longue durée.
}



\bpar{}{

}








%-------------------------


\section{Research Question}

To close this thematic touring introducing chapter, we can state a general research question that frames our further theoretical constructions and first modeling attempts. It is roughly the same as the problematic given at the end of previous section, but adding the insight of modeling as the approach to understand these complex systems.

%networked territorial systems with an emphasize on the role of transportation networks in system evolution processes.

\medskip

\textbf{General research Question.} \textit{To what extent a modeling approach to territorial systems as networked human territories can help disentangling complexly involved processes ?}

\medskip

This question will be refined by theoretical developments in the next chapter and experiments in the followings.






